\section{Running Moose applications}
\label{chap:patterns-running-moose}

\autoref{lst:moose-app-execution} shows the command line often used to execute
a Moose application with an specific input file. This command line uses the
Linux \codeword{time} command to print a summary of the real-time, user CPU
time and system CPU time spent. Furthermore, this command uses the Message
Passing Interface (MPI) via \codeword{mpiexec} to enable parallel processing
across multiple processes.

\begin{lstlisting}[
      language=bash, 
      caption={Executing the dbg-variant of a moose-app},
      label={lst:moose-app-execution},
      escapechar=!,
      belowskip=-0.8 \baselineskip
    ]
!\tikzmark{time0}!time!\tikzmark{time1}! !\tikzmark{mpiexec0}!mpiexec -n 16!\tikzmark{mpiexec1}! !\tikzmark{app0}!./your-app-name-dbg!\tikzmark{app1}! !\tikzmark{inputfile0}!-i ./path-to-input-file.i!\tikzmark{inputfile1}!
!\vspace{8\baselineskip}!
\end{lstlisting}%
\begin{tikzpicture}[remember picture,overlay]

  % draws a curly brace
  \def\drawHorizontalBrace[#1][#2][#3][#4][#5][#6]{%
    \begin{scope}[>=latex] % redef arrow for dimension lines
      \coordinate[yshift=#1] (midpoint34) at ($(#3)!0.5!(#4)$);%
      \coordinate[yshift=-0.3em-0.5em] (midpointBrace) at (midpoint34);%
      \coordinate[yshift=#2] (vertEndPoint) at (midpointBrace);%
      %\filldraw (midpointBrace) circle (3pt);
      %\filldraw (vertEndPoint) circle (3pt);
      \node[draw=none, fit={(#3) (#4)}, inner sep=0pt] (rectg) {};%
      \draw [decorate,decoration={brace,amplitude=0.3em},very thick]%
      ([yshift=#1]rectg.south east) --%
      coordinate[right=0em, midway] (#3#4)%
      ([yshift=#1]rectg.south west);%
      \node[below=0.0em of vertEndPoint, anchor=north west, xshift=#5, align=left] (caption) {#6};%
      \draw (midpointBrace) -- (vertEndPoint);%
    \end{scope}%
  }

  %\filldraw (pic cs:time0) circle (3pt);
  %\filldraw (pic cs:time1) circle (3pt);
  \drawHorizontalBrace[-0.25em][-6.6\baselineskip][pic cs:time0][pic cs:time1][-1em][{Linux 'time' command: print the real-time, user CPU time and system CPU time spent.}]%
  \drawHorizontalBrace[-0.25em][-3.4\baselineskip][pic cs:mpiexec0][pic cs:mpiexec1][-1em][{Execute using MPI to use multiple processes (parallel execution).\\The number determines the count of processes.\\This part can be omitted if only one process is to be used.}]%
  \drawHorizontalBrace[-0.25em][-1.2\baselineskip][pic cs:app0][pic cs:app1][-1em][{Path and name of the Moose application to execute.\\The suffix denotes the variant (here 'dbg' for the debug-variant).}]%
  \drawHorizontalBrace[-0.25em][-0.0\baselineskip][pic cs:inputfile0][pic cs:inputfile1][-4em][{Argument to the Moose application: input file}]%
\end{tikzpicture}

\section{Partial input files}
\label{chap:patterns-partial-input-files}

When a Moose model is defined as an
\href{https://mooseframework.inl.gov/application_usage/input_syntax.html}{‘input
  file’} (usually with the file extension ‘*.i’), the size of such an input file
can quickly grow to several hundred lines, even without the model geometry.
Editing and troubleshooting becomes correspondingly confusing. To mitigate,
Moose supports ‘partial input files’, where other input files can be called up
from a ‘central’ input file. For example, the material parameters or variables
with groups of geometrical objects could be maintained in a separate file.

The recommended option is to use the \codeword{!include} keyword within the
input file. This option is shown in \autoref{patterns-include}. Please note,
that neither path nor filename are allowed to contain a newline character,
\codeword{#} character, or \codeword{[} character.

\begin{lstlisting}[language=Moose, caption={Include another input file},label={patterns-include}]
!include "path/to/file.i"
\end{lstlisting}

The alternative and not recommended option is to call a Moose app with several
input files as arguments. These input files are interpreted by simply
concatenating them. This functionality was included in Moose for test purposes,
but is not suitable for productive models, as this call may not be not
documented and may not be reproducible:

\codeword{./moose-opt -i input1.i -i input2.i -i input3.i}

\section{Physical units}
\label{chap:patterns-physical-units}

Internally, almost all values in Moose are represented purely by numerical
values, without the physical unit also being stored. Accordingly, Moose relies
on the user to specify all numerical values in a consistent system of units in
the input file. For instance, if ‘seconds’ has been selected as the unit of
time and “metres” as the unit of length, all speed values must be specified in
“metres per second”. Without any further action on the part of the user,
neither a check nor a conversion of the units takes place.

In addition, Moose - like other FE programmes - calculates internally with a
finite accuracy. To avoid rounding errors and inaccuracies in the numerical
calculations, it is therefore important that the numerical values entered are
not too large. By carefully selecting and consistently applying the physical
unit system in the input, the user can avoid such numerical problems.

It should also be noted at this point that the choice of physical units
influences other input variables, such as the convergence criteria (see
\autoref{chap:patterns-converge-criteria-and-scaling}), as well as the various
material parameters, initial conditions and boundary conditions. All of these
must be entered in the system of units valid for the model.

From the user's point of view, the challenge lies in entering numerical values
in the ‘correct’ system of physical units everywhere in the input file.
Conversion and verifiability are sources of error that should not be
underestimated, especially in the case of physical units with a time reference.
Experience shows that this is a frequent source of error.

To mitigate this, Moose offers the option of using the
‘\href{https://mooseframework.inl.gov/source/utils/Units.html}{MooseUnits}’
tool in the input files for the explicit input and conversion of physical
units. The input file fragment shown in
\autoref{patterns-physical-units-example1} converts \SI{1000e3}{\kN\per\m^2} to
\SI{}{\MN\per\m^2} and assigns the result to the variable \codeword{E}.

\begin{lstlisting}[language=Moose, caption={Converting \SI{1000e3}{\kN\per\m^2} into
    \SI{}{\MN\per\m^2}},label={patterns-physical-units-example1}]
E = ${units 1000E3 kN/m^2 -> MN/m^2}
\end{lstlisting}

In connection with variables holding the desired physical units for the model
this utility can be used to enforce a consistent system of physical units in
the input file and the user can simultaneously use the physical units of his
source data (for better verifiability).
\autoref{patterns-physical-units-example2} shows the use of a variable used to
hold the desired model unit for lengths.

\begin{lstlisting}[language=Moose, caption={Using a variable to hold the physical unit for lengths},label={patterns-physical-units-example2}]
# variable holding the pyhsical unit for lengths
modelunit_length = 'm'

# conversion of 10 mm into the desired model unit (m)
a = ${units 10 mm -> ${modelunit_length}}
\end{lstlisting}

A full set of pysical units can be setup as shown in
\autoref{patterns-physical-unit-set}. If not all variables are used in the
input file, Moose will issue an error. This can be avoided by adding fake-users
to the variables.

\begin{lstlisting}[language=Moose, caption={Set of consistent physical units in a Moose input file},label={patterns-physical-unit-set}]
# model units
modelunit_length = 'm'
modelunit_time = 's'    #s = seconds, h = hours, day = days
modelunit_mass = 'kg' # Mg = tons; Gg = kilotons

# derived units (this may be moved into a !include)
modelunit_force  = ${raw ${modelunit_mass} * ${modelunit_length} / ${modelunit_time} ^ 2}
modelunit_pressure = ${raw ${modelunit_force} / ${modelunit_length} ^ 2}
modelunit_acceleration = ${raw ${modelunit_length} / ${modelunit_time} ^ 2}
modelunit_density = ${raw ${modelunit_mass} / ${modelunit_length} ^ 3}

# Fake users for the variables above
[Functions]
    [FakeUser_modelunit_force]
        type = ParsedFunction
        expression = 'a'
        symbol_names = 'a'
        symbol_values = '1'
        control_tags = ${modelunit_force}
    []
    (*@{\raisebox{-1pt}[0pt][0pt]{$\vdots$}}@*)
[]
\end{lstlisting}

A selection of model unit combinations can be found in
\autoref{tab:model-units}.

\begin{table}[htbp]
  \centering
  \caption{selection of model unit combinations and derived units}
  \label{tab:model-units}
  \begin{tabularx}{\textwidth}{
      >{\hsize=0.09\hsize\linewidth=\hsize}Y
      >{\hsize=0.09\hsize\linewidth=\hsize}Y
      >{\hsize=0.09\hsize\linewidth=\hsize}Y
      |
      >{\hsize=0.20\hsize\linewidth=\hsize}Y
      >{\hsize=0.20\hsize\linewidth=\hsize}Y
      >{\hsize=0.15\hsize\linewidth=\hsize}Y
      >{\hsize=0.15\hsize\linewidth=\hsize}Y}
    \hline
    Length                                                                             &
    Time                                                                               &
    Mass                                                                               &
    Force                                                                              &
    Pressure                                                                           &
    Acceleration                                                                       &
    Density                                                                              \\

    \hline

    \si[per-mode = symbol]{\metre}                                                     &
    \si[per-mode = symbol]{\second}                                                    &
    \si[per-mode = symbol]{\kilogram}                                                  &
    \si[per-mode = symbol]{\kilogram\cdot\metre\per\second\squared} = \si{\newton}     &
    \si[per-mode = symbol]{\newton\per\metre\squared} \light{= \si{\pascal}}           &
    \si[per-mode = symbol]{\metre\per\second\squared}                                  &
    \si[per-mode = symbol]{\kilogram\per\metre\cubed}                                    \\

    \si[per-mode = symbol]{\metre}                                                     &
    \si[per-mode = symbol]{\second}                                                    &
    \si[per-mode = symbol]{\ton}                                                       &
    \si[per-mode = symbol]{\ton\cdot\metre\per\second\squared} = \si{\kilo\newton}     &
    \si[per-mode = symbol]{\kilo\newton\per\metre\squared} \light{= \si{\kilo\pascal}} &
    \si[per-mode = symbol]{\metre\per\second\squared}                                  &
    \si[per-mode = symbol]{\ton\per\metre\cubed}                                         \\

    \hline
  \end{tabularx}
\end{table}

\section{Moose code plugins}
\label{chap:patterns-code-plugins}

If MooseApps are extended with new MooseObjects that should be shared with
other users, e.g. in workgroups, the need arises to be able to transfer this
functionality to other MooseApps as a plug-in. Preferably without having to
manually copy and adapt many code files. Furthermore, it is often good if there
is independent source code management or versioning. The idea is to realise
such code plugins for MooseApps as git submodules.

To address this need and provide versioning via a git submodule, Moose code
plugins to be found in the following link can be used:
\href{https://github.com/jmeier/mooseapp-code-plugin}{github.com/jmeier/mooseapp-code-plugin}

\section{Convergence criteria and variable scaling}
\label{chap:patterns-converge-criteria-and-scaling}

Due to the fact that the finite element method will always provide approximate
solutions, residuals will remain after each simulation step. To allow Moose to
recognize converge, suitable convergence criteria (e.g. tolerated residuals)
must be set. If these convergence criteria are set too small, moose will appear
not to converge and will need very small time steps to do anything at all. If
set too large, Moose may converge to the wrong result.

Closely related to the selection of tolerated residuals is the scaling of
variables. If only one physical aspect is to be modelled (e.g. mechanics), an
increased tolerated residual is equal to an downscaling of the corresponding
model variables. If several pysical aspects are to be modelled (e.g. mechanics
and flow) variable scaling and adjusting tolerated residuals must be applied in
combination.

It should also be noted that the convergence criteria (e.g. tolerated residuals
and variable scaling) must be set in the context of the chosen system of
physical units (\autoref{chap:patterns-physical-units}).

Large parts of this section are taken from
\href{https://mooseframework.inl.gov/modules/porous_flow/convergence.html}{mooseframework.inl.gov}.
and
\href{https://mooseframework.inl.gov/modules/solid_mechanics/Convergence.html}{mooseframework.inl.gov}

\subsection{\todomarker Nonlinear and (nested) linear solves}

(ToDo)

\subsection{Residuals}

In this subsection data is collected on how to estimate the absolute residuals
to be tolerated (e.g. \texttt{nl\_abs\_tol} of an executioner) for selected
physical aspects.

ToDo: Using \texttt{Debug/show\_var\_residual\_norms = true} the residuals of
the individual variables are printed.

ToDo: Moose will print "Outlier Variable Residual Norms", explained
\href{https://groups.google.com/g/moose-users/c/76wkbH-cuFo/m/OT1Z6vbrm_0J}{here}:
\begin{quote}
  The outliers are those variables whose square of their individual L2-norm is
  "too far" from the square of the average norm. Right now "too far" is some
  fixed value (currently 2.0) determined empirically by a computer scientist ;)
  If you want to see the actually calculation of that value, it's in
  NonlinearSystem.C. We print them because they might be too big to reliably pass
  a tolerance check. It's sort of like a yellow flag if you see a bunch of those
  coming out. You might try tighter tolerances or different solver options, etc .
\end{quote}

\importantbox{
  Residuals of a variable are affected by the scaling of the variable.\newline (\autoref{patterns:variable-scaling}).
}

\subsubsection{Fluid}

The residual for the fluid phase should be a measure of what is still
considered a \emph{steady state} in the pore pressure field. This residual for
the fluid phase can be estimated using \autoref{patterns:eq:residual-fluid},
where beside some material constants for the fluid (density $\rho_\mathrm{f}$
and dynamic viscosity $\mu_\mathrm{f}$) and the porosity $\kappa$, the accepted
pressure gradient $\epsilon_\mathrm{f}$ (e.g. $\epsilon_\mathrm{f} =
  \qty[per-mode = symbol]{1}{\newton\per\square\metre\per\metre} $) must be
definded for a volume of interest $V$. In geotechnics, this volume of interest
is often only a part of the full mesh.

\begin{equation}
  \label{patterns:eq:residual-fluid}
  R_{f} \approx V \cdot \det (\kappa) \cdot \epsilon_\mathrm{f} \cdot \rho_\mathrm{f} / \mu_\mathrm{f}
\end{equation}
with:
\begin{description}
  \item[$R_\mathrm{f}$] estimate of residual for the fluid
  \item[$V$] Volume of interest (physical unit: volume, e.g \unit{\cubic\metre})
  \item[$\kappa$] tensor of permeabilities (physical unit of the tensor items: area, e.g. \unit{\square\metre})
  \item[$\epsilon_\mathrm{f}$] desired precision (physical unit: pressure per length, e.g. \unit[per-mode = symbol]{\pascal\per\metre} = \unit[per-mode = symbol]{\newton\per(\square\metre\cdot\metre)} )
  \item[$\rho_\mathrm{f}$] fluid density (physical unit: mass per volume, e.g. \unit[per-mode = symbol]{\kilogram\per\cubic\metre})
  \item[$\mu_\mathrm{f}$] fluid dynamic viscosity (physical unit: mass per length and time, e.g. \unit[per-mode = symbol]{\kilogram\per({\metre\cdot\second})} = \unit[per-mode = symbol]{\pascal\cdot\second})
\end{description}

\vspace{1em}

For water ($\rho_\mathrm{f} \approx \qty[per-mode =
    symbol]{1000}{\kilogram\per\cubic\metre}$, $\mu_\mathrm{f} \approx
  \qty[per-mode = symbol]{1}{\milli\pascal\per\second}$) this equation simplifies
to:

\begin{equation}
  R_{water} \approx V \cdot \det (\kappa) \cdot \epsilon_\mathrm{water} \cdot \SI{E-6}{(\kilogram\cdot\second)\per(\cubic\metre\cdot\pascal)}
\end{equation}

\subsubsection{Mechanics}

Very similar to the residuals of the other phases, the residual for the solid
phase should be a measure of what is still considered a \emph{steady state} in
the stress field. This residual for the solid phase can be estimated using
\autoref{patterns:eq:residual-solid}, again using a volume of interest $V$ and
an acceptable residual stress gradient $\epsilon_\mathrm{s}$ (e.g. the error
accepted in $\nabla \sigma$).

\begin{equation}
  \label{patterns:eq:residual-solid}
  R_{s} \approx V \cdot \left|\epsilon_\mathrm{s}\right|
\end{equation}
with:
\begin{description}
  \item[$R_\mathrm{s}$] estimate of residual for the solid (physical unit: force, e.g. \unit{\newton})
  \item[$V$] Volume of interest (physical unit: volume, e.g \unit{\cubic\metre})
  \item[$\epsilon_\mathrm{s}$] desired precision (physical unit: pressure per length, e.g. \unit[per-mode = symbol]{\pascal\per\metre} = \unit[per-mode = symbol]{\newton\per(\square\metre\cdot\metre)} )
\end{description}

\subsection{Variable scaling}
\label{patterns:variable-scaling}

To allow efficient nonlinear and linear solves, the Jacobians of the different
physics of a Moose model should have comparable magnitudes. Ideally, these
should all be as close to unity as possible. With the possibility of variable
scaling, Moose offers a tool that allows this to be achieved.

Variable scaling primarily affects the internal calculation processes (e.g.
Jacobian and residuals). The simulation results, however, are generally not
scaled, which means that variable scaling does not need to be considered in
post-processing.

\subsubsection{Manual variable scaling}

When defining the variables, the ‘scaling’ property can be used to specify a
manual scaling factor for each variable individually.
\autoref{patterns-variable-scaling-code} shows a snipped of an input file with
a scaling factor of \qty{1E+5}{} applied to the variable ‘porepressure’.

\begin{lstlisting}[language=Moose, caption={Variable definition with scaling on porepressure in a Moose inut file},label={patterns-variable-scaling-code}]
[Variables]
    [disp_x]
        family = LAGRANGE
        order = SECOND
    []
    [disp_y]
        family = LAGRANGE
        order = SECOND
    []
    [disp_z]
        family = LAGRANGE
        order = SECOND
    []
    [porepressure]
        family = LAGRANGE
        order = SECOND
        scaling = 1E+5
    []
[]
\end{lstlisting}

Manual variable scaling can be determined as follows:
\begin{enumerate}
  \item Variable scaling of all variables is set to unity.
  \item\label{ite:patterns:variable-scaling:manual:test_run_residuals} {Using
          \texttt{Debug/show\_var\_residual\_norms = true} the residuals of the
          individual variables are printed for a test run of the model. The residuals for
          the first time step ("Time Step 1") may look like the following example:
          \vspace{0.1cm}\begin{mdframed}[leftmargin=0.1cm,innerleftmargin=0.1cm,topline=false,rightline=false,bottomline=false]
            \begin{verbatim}
|residual|_2 of individual variables:
          disp_x:       19.4404
          disp_y:       19.4404
          disp_z:       110.432
          porepressure: 7.33977e-05
0 Nonlinear |R| = 1.138024e+02\end{verbatim}
          \end{mdframed}}
  \item In the example in
        \autoref{ite:patterns:variable-scaling:manual:test_run_residuals}, the
        magnitude of the residuals for the \texttt{disp}-variables are roughly the
        same, while the magnitude of the \texttt{porepressure} is much smaller.
        Therefore, without scaling, the contribution to the residual of the
        \texttt{porepressure} would be strongly underrepresented and one would need
        very small \texttt{abs\_norm\_tol} to make sure that the hydraulic equation
        converges. On the other hand, this may cause convergence problems for the
        displacement equations as they have larger residuals.\newline It is therefore
        advisable for this model to scale the \texttt{porepressure} in such a way that
        the associated residuals are comparable with those of the \texttt{disp}
        variables using (while leaving scaling of all \texttt{disp} variables at
        unity): \texttt{Variables/porepressure/scaling = 1E+5}

\end{enumerate}

\subsubsection{Automatic variable scaling}

As alternative to manual variable scaling, Moose offers the functionality of
automatic variable scaling, calculating the scaling factors automatically and
overriding any - if present - manually given scaling factors. Automatic scaling
can be activated by setting \texttt{automatic\_scaling = true} in the
\texttt{[Executioner]} block. To output the calculated scaling factors, set
\texttt{verbose = true} in the \texttt{[Executioner]} block.

In the author's experience, at least for hydro-mechanically coupled models,
automatic scaling often leads to suboptimal convergence compared to manual
scaling. Therefore, it is recommended to use manual scaling instead.

\subsection{Tolerances and their default values}

\begin{table}[htbp]
  \centering
  \caption{Default values for tolerances and maximum number of iterations}
  \label{tab:tolerances}
  \begin{tabularx}{\textwidth}{
      >{\hsize=0.10\hsize\linewidth=\hsize}Y
      >{\hsize=0.30\hsize\linewidth=\hsize}Y
      >{\hsize=0.30\hsize\linewidth=\hsize}Y
      >{\hsize=0.30\hsize\linewidth=\hsize}Y}
    \hline
    solve     & absolute tolerance            & relative tolerance            & maximum iterations \\

    \hline

    linear    & \texttt{l\_abs\_tol = 1e-50}  & \texttt{l\_tol = 1e-05}       &
    \texttt{l\_max\_its = 10000}                                                                   \\

    nonlinear & \texttt{nl\_abs\_tol = 1e-50} & \texttt{nl\_rel\_tol = 1e-08} &
    \texttt{nl\_max\_its = 50}                                                                     \\

    \hline
  \end{tabularx}
\end{table}

Physical meaning of a (absolute) tolerance depends on the variable the
tolerance is definded for. Solid mechanics displacement variables (often called
\texttt{disp\_x}, \texttt{disp\_y}, and \texttt{disp\_z}) is:

$ stress \cdot strain = \unit[per-mode = symbol]{\kN\per\square\metre} \cdot \unit[per-mode = symbol]{\metre\per\metre} = \frac{Energy}{Volume} = \unit[per-mode = symbol]{\kilo\newton} \cdot \unit[per-mode = symbol]{\metre\per\cubic\metre} $

Some collected notes on the tolerances and how to choose:

\begin{itemize}
  \item Assuming the residual close to unity, \texttt{nl\_abs\_tol} and
        \texttt{l\_abs\_tol} should be larger than 1e-15 to be not too close to
        numerical precision.
  \item The linear tolerance should be looser than the nonlinear tolerance (the linear
        tolerance should be a larger number than for the nonlinear). In fact since the
        residual is checked by the nonlinear solver, one don't need to converge the
        linear solve anywhere near as tight.
  \item From experience, \texttt{l\_tol = 1e-3} is often enough.
  \item According to
        \href{https://petsc.org/release/manualpages/KSP/KSPConvergedDefault/}{petsc.org},
        the default convergence algorithm reaches convergence when both (absolute and
        relative) tolerances are satisfied and the following equation holds: \newline
        $rnorm < \max (\texttt{rel\_tol} \cdot rnorm_0, \texttt{abs\_tol})$.
  \item Relative tolerances are often chosen to be \SI{1}{\percent} in other
        geotechnical finite element codes.
\end{itemize}

% https://mooseframework.inl.gov/moose/application_usage/failed_solves.html
% convergence problems:

% (1) switch to Newton instead of PJFNK

% (2) use the jacobian analyzer to make sure your jacobian is correct
% for *small* problems, one may add "-snes_test_jacobian" in "petsc_options" of [Preconditioning/SMP]

% (3) add a strong preconditioner like LU
%   petsc_options_iname = '-pc_type -pc_factor_mat_solver_package'
%   petsc_options_value = ' lu       mumps'

% (4) check if the problem is singular using SVD

% on tolerances:
%    https://github.com/idaholab/moose/discussions/27576#discussioncomment-9354227
%    I don't usually say this but your convergence criteria are too tight
%      nl_abs_tol = 1E-10
%      l_abs_tol = 1e-15
%    you can leave the first one if it works, but the second one is too close to 
%    numerical precision. In fact since the residual is checked by the nonlinear 
%    solver, you don't need to converge the linear solve anywhere near as tight. 
%    1e-3 is often enough
%
%    https://github.com/idaholab/moose/discussions/26115#discussioncomment-7634493
%    - The linear tolerance should be looser than the nonlinear tolerance; 
%      it should be a larger number than for the nonlinear