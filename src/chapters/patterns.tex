\section{Partial input files}
\label{chap:patterns-partial-input-files}

When a Moose model is defined as an
\href{https://mooseframework.inl.gov/application_usage/input_syntax.html}{‘input
    file’} (usually with the file extension ‘*.i’), the size of such an input file
can quickly grow to several hundred lines, even without the model geometry.
Editing and troubleshooting becomes correspondingly confusing. Moose therefore
also supports ‘partial input files’, where other input files can be called up
from a ‘central’ input file. For example, the material parameters or variables
with groups of geometrical objects could be maintained in a separate file.

The recommented option is to use the \codeword{!include} keyword within the
input file. This option is shown in \autoref{patterns-include}. Please note,
that the neither path not filename are allowed to contain a newline character,
\codeword{#} character, or \codeword{[} character.

\begin{lstlisting}[language=perl, caption={Include anpther input file},label={patterns-include}]
!include "path/to/file.i"
\end{lstlisting}

The alternative and not recommented option is to call a Moose app with several
input files as arguments. These input files are interpreted by simply
concatenating them. This functionality was included in Moose for test purposes,
but is not suitable for productive models, as this call may not be not
documented and may not be reproducible:

\codeword{./moose-opt -i input1.i -i input2.i -i input3.i}

\section{Physical units}
\label{chap:patterns-physical-units}

In Moose, all numerical values must be entered in a system of units which is
consistent in itself - but which is selected by the user. Without additional
options, the user must enter a number in the input file without a unit being
explicitly labelled.

In addition, Moose - like other FE programmes - calculates internally with a
finite precision. To avoid rounding errors and inaccuracies in the numerical
calculation, it is therefore important that the numerical values entered are
not too large. With a careful choice of the system of physical units, the user
can avoid such numerical problems.

It is important to emphasise at this point that, in addition to the various
material parameters, initial conditions and boundary conditions, the
convergence criteria are also affected by the choice of physical units (see
\autoref{chap:patterns-converge-criteria-and-scaling}).

From the user's point of view, the challenge is to enter the numerical values
in the ‘correct’ system of physical units everywhere in the input file. This
conversion and the non-trivial verifiability is a source of error that should
not be underestimated, particularly in the case of physical units with a time
reference. Experience shows, that this is a frequent source of problems!

To mitigate this source of error, Moose offers the
‘\href{https://mooseframework.inl.gov/source/utils/Units.html}{MooseUnits}’
unitily. In the input file this utility allows to convert the unit ‘on the
fly’. The input file fragment shown in
\autoref{patterns-physical-units-example1} converts \SI{1000e3}{\kN\per\m^2}
into \SI{}{\MN\per\m^2} and assigns the result to the variable \codeword{E}.

\begin{lstlisting}[language=perl, caption={Converting \SI{1000e3}{\kN\per\m^2} into
    \SI{}{\MN\per\m^2}},label={patterns-physical-units-example1}]
E = ${units 1000E3 kN/m^2 -> MN/m^2}
\end{lstlisting}

In connection with variables holding the desired physical units for the model
this utility can be used to enforce a consistent system of physical units in
the input file and the user can simultaneously use the physical units of his
source data (for better verifiability).
\autoref{patterns-physical-units-example2} shows the use of a variable used to
hold the desired model unit for lengths.

\begin{lstlisting}[language=perl, caption={Using a variable to hold the physical unit for lengths},label={patterns-physical-units-example2}]
# variable holding the pyhsical unit for lengths
modelunit_length = 'm'

# conversion of 10 mm into the desired model unit (m)
a = ${units 10 mm -> ${modelunit_length}}
\end{lstlisting}

A full set of pysical units can be setup as shown in
\autoref{patterns-physical-unit-set}. If not all variables are used in the
input file, Moose will issue an error. This can be avoided by adding fake-users
to the variables.

\begin{lstlisting}[language=perl, caption={Set of consistent physical units in a Moose inut file},label={patterns-physical-unit-set}]
# model units
modelunit_length = 'm'
modelunit_time = 's'    #s = seconds, h = hours, day = days
modelunit_mass = 'kg' # Mg = tons; Gg = kilotons

# derived units (this may be moved into a !include)
modelunit_force  = ${raw ${modelunit_mass} * ${modelunit_length} / ${modelunit_time} ^ 2}
modelunit_pressure = ${raw ${modelunit_force} / ${modelunit_length} ^ 2}
modelunit_acceleration = ${raw ${modelunit_length} / ${modelunit_time} ^ 2}
modelunit_density = ${raw ${modelunit_mass} / ${modelunit_length} ^ 3}

# Fake userss for the variables above
[Functions]
    [FakeUser_modelunit_force]
        type = ParsedFunction
        expression = 'a'
        symbol_names = 'a'
        symbol_values = '1'
        control_tags = ${modelunit_force}
    []
    (*@{\raisebox{-1pt}[0pt][0pt]{$\vdots$}}@*)
[]
\end{lstlisting}

A selection of model unit combinations can be found in
\autoref{tab:model-units}.

\begin{table}[htbp]
    \centering
    \caption{selection of model unit combinations and derived units}
    \label{tab:model-units}
    \begin{tabularx}{\textwidth}{
            >{\hsize=0.10\hsize\linewidth=\hsize}Y
            >{\hsize=0.10\hsize\linewidth=\hsize}Y
            >{\hsize=0.10\hsize\linewidth=\hsize}Y
            |
            >{\hsize=0.20\hsize\linewidth=\hsize}Y
            >{\hsize=0.20\hsize\linewidth=\hsize}Y
            >{\hsize=0.15\hsize\linewidth=\hsize}Y
            >{\hsize=0.15\hsize\linewidth=\hsize}Y}
        \hline
        Length                                                                             &
        Time                                                                               &
        Mass                                                                               &
        Force                                                                              &
        Pressure                                                                           &
        Acceleration                                                                       &
        Density                                                                              \\

        \hline

        \si[per-mode = symbol]{\metre}                                                     &
        \si[per-mode = symbol]{\second}                                                    &
        \si[per-mode = symbol]{\kilogram}                                                  &
        \si[per-mode = symbol]{\kilogram\cdot\metre\per\second\squared} = \si{\newton}     &
        \si[per-mode = symbol]{\newton\per\metre\squared} \light{= \si{\pascal}}           &
        \si[per-mode = symbol]{\metre\per\second\squared}                                  &
        \si[per-mode = symbol]{\kilogram\per\metre\cubed}                                    \\

        \si[per-mode = symbol]{\metre}                                                     &
        \si[per-mode = symbol]{\second}                                                    &
        \si[per-mode = symbol]{\ton}                                                       &
        \si[per-mode = symbol]{\ton\cdot\metre\per\second\squared} = \si{\kilo\newton}     &
        \si[per-mode = symbol]{\kilo\newton\per\metre\squared} \light{= \si{\kilo\pascal}} &
        \si[per-mode = symbol]{\metre\per\second\squared}                                  &
        \si[per-mode = symbol]{\ton\per\metre\cubed}                                         \\

        \hline
    \end{tabularx}
\end{table}

\section{Moose code plugins}
\label{chap:patterns-code-plugins}

If MooseApps are extended with new MooseObjects that should be shared with
other users, e.g. in workgroups, the need arises to be able to transfer this
functionality to other MooseApps as a plug-in. Preferably without having to
manually copy and adapt many code files. Furthermore, it is often good if there
is independent source code management or versioning. The idea is to realise
such code plugins for MooseApps as git submodules.

To address this need and provide versioning via a git submodule, Moose code
plugins to be found in the following link can be used:
\href{https://github.com/jmeier/mooseapp-code-plugin}{github.com/jmeier/mooseapp-code-plugin}

\section{Convergence criteria and variable scaling}
\label{chap:patterns-converge-criteria-and-scaling}

Due to the fact that the finite element method will always provide approximate
solutions, residuals will remain after each simulation step. To allow Moose to
recognize converge, suitable convergence criteria (e.g. tolerated residuals)
must be set. If these convergence criteria are set too small, the moose will
appear not to converge and will need very small time steps to do anything at
all. If set too large, Moose may converge to the wrong result.

Closely related to the selection of tolerated residuals is the scaling of
variables. If only one physical aspect is to be modelled (e.g. mechanics), an
increased tolerated residual is equal to an downscaling of the corresponding
model variables. If several pysical aspects are to be modelled (e.g. mechanics
and flow) variable scaling and adjusting tolerated residuals must be applied in
combination.

It should also be noted that the convergence criteria (e.g. tolerated residuals
and variable scaling) must be set in the context of the chosen system of
physical units (\autoref{chap:patterns-physical-units}).

Large parts of this section is taken from
\href{https://mooseframework.inl.gov/modules/porous_flow/convergence.html}{mooseframework.inl.gov}.

\subsection{Residuals}

In this subsection data is collected on how to estimate the absolute residuals
to be tolerated (e.g. \texttt{-nl\_abs\_tol}) for selected physical aspects.

\subsubsection{Fluid}

The residual for the fluid phase should be a measure of what is still
considered a \emph{steady state} in the pore pressure field. This residual for
the fluid phase can be estimated using \autoref{patterns:eq:residual-fluid},
where beside some material constants for the fluid (density $\rho_\mathrm{f}$
and dynamic viscosity $\mu_\mathrm{f}$) and the porosity $\kappa$, the accepted
pressure gradient $\epsilon_\mathrm{f}$ (e.g. $\epsilon_\mathrm{f} =
    \qty[per-mode = symbol]{1}{\newton\per\square\metre\per\metre} $) must be
definded for a volume of interest $V$. In geotechnics, this volume of interest
is often only a part of the full mesh.

\begin{equation}
    \label{patterns:eq:residual-fluid}
    R_{f} \approx V \cdot \det (\kappa) \cdot \epsilon_\mathrm{f} \cdot \rho_\mathrm{f} / \mu_\mathrm{f}
\end{equation}
with:
\begin{description}
    \item[$R_\mathrm{f}$] estimate of residual for the fluid
    \item[$V$] Volume of interest (physical unit: volume, e.g \unit{\cubic\metre})
    \item[$\kappa$] tensor of permeabilities (physical unit of the tensor items: area, e.g. \unit{\square\metre})
    \item[$\epsilon_\mathrm{f}$] desired precision (physical unit: pressure per length, e.g. \unit[per-mode = symbol]{\pascal\per\metre} = \unit[per-mode = symbol]{\newton\per(\square\metre\cdot\metre)} )
    \item[$\rho_\mathrm{f}$] fluid density (physical unit: mass per volume, e.g. \unit[per-mode = symbol]{\kilogram\per\cubic\metre})
    \item[$\mu_\mathrm{f}$] fluid dynamic viscosity (physical unit: mass per length and time, e.g. \unit[per-mode = symbol]{\kilogram\per({\metre\cdot\second})} = \unit[per-mode = symbol]{\pascal\cdot\second})
\end{description}

\vspace{1em}

For water ($\rho_\mathrm{f} \approx \qty[per-mode =
        symbol]{1000}{\kilogram\per\cubic\metre}$, $\mu_\mathrm{f} \approx
    \qty[per-mode = symbol]{1}{\milli\pascal\per\second}$) this equation simplifies
to:

\begin{equation}
    R_{water} \approx V \cdot \det (\kappa) \cdot \epsilon_\mathrm{water} \cdot \SI{E-6}{(\kilogram\cdot\second)\per(\cubic\metre\cdot\pascal)}
\end{equation}

\subsubsection{Mechanics}

Very similar to the residuals of the other phases, the residual for the solid
phase should be a measure of what is still considered a \emph{steady state} in
the stress field. This residual for the solid phase can be estimated using
\autoref{patterns:eq:residual-solid}, again using a volume of interest $V$ and
an acceptable residual stress gradient $\epsilon_\mathrm{s}$ (e.g. the error
accepted in $\nabla \sigma$).

\begin{equation}
    \label{patterns:eq:residual-solid}
    R_{s} \approx V \cdot \left|\epsilon_\mathrm{s}\right|
\end{equation}
with:
\begin{description}
    \item[$R_\mathrm{s}$] estimate of residual for the solid (physical unit: force, e.g. \unit{\newton})
    \item[$V$] Volume of interest (physical unit: volume, e.g \unit{\cubic\metre})
    \item[$\epsilon_\mathrm{s}$] desired precision (physical unit: pressure per length, e.g. \unit[per-mode = symbol]{\pascal\per\metre} = \unit[per-mode = symbol]{\newton\per(\square\metre\cdot\metre)} )
\end{description}

\subsection{Variable scaling}

\texttt{[Executioner]} \texttt{automatic\_scaling = true}

\begin{lstlisting}[language=perl, caption={Variable definition with scaling on porepressure in a Moose inut file},label={patterns-variable-scaling-code}]
[Variables]
    [disp_x]
        family = LAGRANGE
        order = SECOND
    []
    [disp_y]
        family = LAGRANGE
        order = SECOND
    []
    [disp_z]
        family = LAGRANGE
        order = SECOND
    []
    [porepressure]
        family = LAGRANGE
        order = SECOND
        scaling = 1E-8
    []
[]
\end{lstlisting}

% https://mooseframework.inl.gov/moose/application_usage/failed_solves.html
% convergence problems:

% (1) switch to Newton instead of PJFNK

% (2) use the jacobian analyzer to make sure your jacobian is correct
% for *small* problems, one may add "-snes_test_jacobian" in "petsc_options" of [Preconditioning/SMP]

% (3) add a strong preconditioner like LU
%   petsc_options_iname = '-pc_type -pc_factor_mat_solver_package'
%   petsc_options_value = ' lu       mumps'

% (4) check if the problem is singular using SVD

% on tolerances:
%    https://github.com/idaholab/moose/discussions/27576#discussioncomment-9354227
%    Hello
%    
%    I don't usually say this but your convergence criteria are too tight
%      nl_abs_tol = 1E-10
%      l_abs_tol = 1e-15
%    you can leave the first one if it works, but the second one is too close to 
%    numerical precision. In fact since the residual is checked by the nonlinear 
%    solver, you don't need to converge the linear solve anywhere near as tight. 
%    1e-3 is often enough
%    
%    Guillaume