\section{Partial input files}
\label{chap:patterns-partial-input-files}

When a Moose model is defined as an
\href{https://mooseframework.inl.gov/application_usage/input_syntax.html}{‘input
    file’} (usually with the file extension ‘*.i’), the size of such an input file
can quickly grow to several hundred lines, even without the model geometry.
Editing and troubleshooting becomes correspondingly confusing. Moose therefore
also supports ‘partial input files’, where other input files can be called up
from a ‘central’ input file. For example, the material parameters or variables
with groups of geometrical objects could be maintained in a separate file.

The recommented option is to use the \codeword{!include} keyword within the
input file. This option is shown in \autoref{patterns-include}. Please note,
that the neither path not filename are allowed to contain a newline character,
\codeword{#} character, or \codeword{[} character.

\begin{lstlisting}[language=perl, caption={Include anpther input file},label={patterns-include}]
!include "path/to/file.i"
\end{lstlisting}

The alternative and not recommented option is to call a Moose app with several
input files as arguments. These input files are interpreted by simply
concatenating them. This functionality was included in Moose for test purposes,
but is not suitable for productive models, as this call may not be not
documented and may not be reproducible:

\codeword{./moose-opt -i input1.i -i input2.i -i input3.i}

\section{Physical units}
\label{chap:patterns-physical-units}

In Moose, all numerical values must be entered in a system of units which is
consistent in itself - but which is selected by the user. Without additional
options, the user must enter a number in the input file without a unit being
explicitly labelled.

In addition, Moose - like other FE programmes - calculates internally with a
finite precision. To avoid rounding errors and inaccuracies in the numerical
calculation, it is therefore important that the numerical values entered are
not too large. With a careful choice of the system of physical units, the user
can avoid such numerical problems.

From the user's point of view, the challenge is to enter the numerical values
in the ‘correct’ system of physical units everywhere in the input file. This
conversion and the non-trivial verifiability is a source of error that should
not be underestimated, particularly in the case of physical units with a time
reference. Experience shows, that this is a frequent source of problems!

To mitigate this source of error, Moose offers the
‘\href{https://mooseframework.inl.gov/source/utils/Units.html}{MooseUnits}’
unitily. In the input file this utility allows to convert the unit ‘on the
fly’. The input file fragment shown in
\autoref{patterns-physical-units-example1} converts \SI{1000e3}{\kN\per\m^2}
into \SI{}{\MN\per\m^2} and assigns the result to the variable \codeword{E}.

\begin{lstlisting}[language=perl, caption={Converting \SI{1000e3}{\kN\per\m^2} into
    \SI{}{\MN\per\m^2}},label={patterns-physical-units-example1}]
E = ${units 1000E3 kN/m^2 -> MN/m^2}
\end{lstlisting}

In connection with variables holding the desired physical units for the model
this utility can be used to enforce a consistent system of physical units in
the input file and the user can simultaneously use the physical units of his
source data (for better verifiability).
\autoref{patterns-physical-units-example2} shows the use of a variable used to
hold the desired model unit for lengths.

\begin{lstlisting}[language=perl, caption={Using a variable to hold the physical unit for lengths},label={patterns-physical-units-example2}]
# variable holding the pyhsical unit for lengths
modelunit_length = 'm'

# conversion of 10 mm into the desired model unit (m)
a = ${units 10 mm -> ${modelunit_length}}
\end{lstlisting}

A full set of pysical units can be setup as shown in
\autoref{patterns-physical-unit-set}. If not all variables are used in the
input file, Moose will issue an error. This can be avoided by adding fake-users
to the variables.

\begin{lstlisting}[language=perl, caption={Set of consistent physical units in a Moose inut file},label={patterns-physical-unit-set}]
    # model units
    modelunit_length = 'm'
    modelunit_time   = 's'
    modelunit_mass = 'kg' # Mg = tons; Gg = kilotons

    # derived units (this may be moved into a !include)
    modelunit_force  = ${raw ${modelunit_mass} * ${modelunit_length} / ${modelunit_time} ^ 2}
    modelunit_pressure = ${raw ${modelunit_force} / ${modelunit_length} ^ 2}
    modelunit_acceleration = ${raw ${modelunit_length} / ${modelunit_time} ^ 2}
    modelunit_density = ${raw ${modelunit_mass} / ${modelunit_length} ^ 3}
\end{lstlisting}