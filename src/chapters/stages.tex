\section{General comments on construction stages}
\label{chap:stages-general}

The type of geotechnical simulation primarily discussed in this document aims
to model a construction process. Accordingly, such a simulation must reproduce
the various stages of construction over time. Since this document assumes that
Moose models are organised on the basis of subdomains
(\autoref{geometry-blocks-and-subdomains}), the state changes also refer
primarily to adjustments that affect entire subdomains.

The most common state changes in this context are:

\begin{itemize}
    \item Activating subdomains (e.g. installation of sheet piles is usually modelled by
          activating the corresponding subdomain)
    \item Deactivating subdomains (e.g. excavation of a soil volume is usually modelled
          by deactivating the corresponding subdomain)
    \item Replacing subdomains (e.g. installation of a slurry wall modelled via volume
          elements leads to the need to replace the soil volume with the volume
          representing the slurry wall)
    \item Adjusting external loads (e.g. loads are used to model various influences. As
          these influences change over time, the corresponding loads have to be adjusted)
    \item Adjusting prestress (e.g. for props and stand anchors the prestress is to be
          adjusted)
\end{itemize}

As a consequence of the Moose input file concept, these status adjustments must
be triggered at various points in an input file if Moose is used ‘out of the
box’. To provide a more centralised way of defining construction stages, the
\codeword{[Stages]} MooseApp code plugin linked below can be used:

\href{https://github.com/jmeier/moose-codeplugin-stages}{github.com/jmeier/moose-codeplugin-stages}

This document primarily describes the procedure for defining construction
stages with the \codeword{[Stages]} code plugin where applicable. The procedure
without this plugin is only mentioned in selected places.

\section{Activating and deactivating subdomains}
\label{chap:stages-subdomain-activation-deactivation}

ToDo: Ramp up / ramp down material properties

(to be written)

\section{Replacing subdomains}
\label{chap:stages-subdomain-replacement}

(to be written)

\section{Adjusting external loads}
\label{chap:stages-loads}

(to be written)

\section{Adjusting prestress}
\label{chap:stages-prestress}

(to be written)