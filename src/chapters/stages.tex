\section{General comments on construction stages}
\label{chap:stages-general}

The type of geotechnical simulation primarily discussed in this document aims
to model a construction process. Accordingly, such a simulation must reproduce
the various stages of construction over time. Since this document assumes that
Moose models are organised on the basis of subdomains
(\autoref{geometry-blocks-and-subdomains}), the state changes also refer
primarily to adjustments that affect entire subdomains.

The most common state changes in this context are:

\begin{itemize}
      \item Activating FE-elements (e.g. installation of sheet piles is usually modelled by
            activating the corresponding set ot FE-elements,
            \autoref{chap:stages-element-activation-deactivation})
      \item Deactivating FE-elements (e.g. excavation of a soil volume is usually modelled
            by deactivating the corresponding set of FE-elements,
            \autoref{chap:stages-element-activation-deactivation})
      \item Replacing FE-elements (e.g. installation of a slurry wall modelled via volume
            elements leads to the need to replace the soil volume with the volume
            representing the slurry wall, \autoref{chap:stages-element-replacement})
      \item Adjusting external loads (e.g. loads are used to model various influences. As
            these influences change over time, the corresponding loads have to be adjusted,
            \autoref{chap:stages-loads})
      \item Adjusting prestress (e.g. for props and stand anchors the prestress is to be
            adjusted, \autoref{chap:stages-prestress})
\end{itemize}

As a consequence of the Moose input file concept, these status adjustments must
be triggered at various points in an input file if Moose is used ‘out of the
box’. To provide a more centralised way of defining construction stages, the
\codeword{[Stages]} MooseApp code plugin linked below can be used:

\href{https://github.com/jmeier/moose-codeplugin-stages}{github.com/jmeier/moose-codeplugin-stages}

This document primarily describes the procedure for the definition of
construction stages, with the use of the \codeword{[Stages]} code plugin where
applicable. The procedure without this plugin is only mentioned in selected
places.

\section{Apprupt vs. gradual changes}
\label{chap:stages-gradual-changes}

For transient simulations, the speed at which changes are applied in the model
usually matter.

(to be written)

\section{Activating and deactivating FE-elements}
\label{chap:stages-element-activation-deactivation}

As mentioned in \autoref{geometry-blocks-and-subdomains}, Moose models are
organized and managed by means of subdomains. By assigning FE elements and
materials to a subdomain, for example, the behaviour of these FE elements is
defined. However, Moose does not offer the option of activating or deactivating
FE elements or subdomains during a simulation run. However, Moose allows FE
elements to be reassigned from one subdomain to another during a simulation run
using the
\href{https://mooseframework.inl.gov/syntax/MeshModifiers/index.html}{MeshModifiers}
system.

On the other hand, Moose allows for subdomains to exist without kernels,
variables, materials and other Moose objects, but for FE elements to be present
in these subdomains. FE-elements in those subdomains ‘without physics’ will
simply be ignored my Moose.

Activation and deactivation of FE elements can thus be modelled by combining
the option to have subdomains ‘without physics’ and the option to reassign FE
elements to another subdomain. For deactivation FE elements are moved from a
subdomain ‘with physics’ to a subdomain ‘without physics’. For activation vice
versa.

ToDo:
\begin{itemize}
      \item Activation / Deactivation with [Stages]
      \item Ramp up / ramp down material properties, \autoref{chap:stages-gradual-changes}
\end{itemize}

\section{Replacing FE-elements}
\label{chap:stages-element-replacement}

(to be written)

\section{Adjusting external loads}
\label{chap:stages-loads}

(to be written)

\section{Adjusting prestress}
\label{chap:stages-prestress}

(to be written)