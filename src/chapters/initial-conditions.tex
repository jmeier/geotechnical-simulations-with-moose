\section{Initial conditions for Moose objects}
\label{chap:IC-moose-objects}

For definition of the initial (starting) conditions for the variables a Moose
simulation the \href{https://mooseframework.inl.gov/syntax/ICs}{ICs system} is
to be used.

(to be written)

\section{Gravitation and initial stress state}
\label{chap:IC-stress-state}

The initial stress state of geotechnical simulations is often non-trivial. This
is because the section of geosphere to be modelled often has a history of
millions of years under different loading conditions, including gravity and
tectonical processes. In addition, the Earth's surface is often non-horizontal,
so that stress trajectories near the surface are oriented accordingly.
Furthermore, the initial stress state may be anisotropic with a specific
orientation. Gravity and the initial stress state are normally taken into
account by means of:

\begin{itemize}
    \item Definition of appropriate displacement boundary conditions for the model (see
          \autoref{chap:model-configuration-boundary-fixities}).
    \item Activation of gravitational body force (see
          \autoref{chap:IC-stress-state-gravity}).
    \item Definition of eigenstrains from the inital stress field (e.g. using
          \href{https://mooseframework.inl.gov/source/materials/ComputeEigenstrainFromInitialStress.html}{ComputeEigenstrainFromInitialStress},
          see \autoref{chap:IC-stress-state-simple} and
          \autoref{chap:IC-stress-state-anisotropic}).
\end{itemize}

\subsection{Gravity}
\label{chap:IC-stress-state-gravity}

To consider gravity, the following aspects must be included in the model:
\begin{itemize}
    \item When compiling the MooseApp: As the effect of gravity is taken into account in
          interaction with the material density, the Moose module \codeword{MISC} must be
          active in addition to the Moose module \codeword{SOLID_MECHANICS}.
    \item Gravity must be activated in the input file. This is done by adding a special
          kernel of the type \codeword{Gravity} as shown in
          \autoref{initial-conditions-gravity}. The direction and strength of the
          gravitational field is to be defined here.
    \item The materials must be assigned a material density
          (\autoref{initial-conditions-density}).
\end{itemize}

\begin{lstlisting}[language=perl, caption={Gravity kernel in a Moose inut file},label={initial-conditions-gravity}]
[Kernels]
    [gravity]
        type = Gravity
        use_displaced_mesh = false
        variable = disp_z
        value = -9.81
    []
[]
\end{lstlisting}

\begin{lstlisting}[language=perl, caption={Assignment of a density to subdomain ‘block1’},label={initial-conditions-density}]
[Materials]
    [undrained_density]
      type = GenericConstantMaterial
      block = 'Block1'
      prop_names = density
      prop_values = 0.0025
    []
[]
\end{lstlisting}

{\hfuzz=20pt
\subsection{Initial stress state using ComputeEigenstrainFromInitialStress}
}
\label{chap:IC-stress-state-simple}

This initial stress state corresponds to the ‘geostatic stress state’ in the
uninfluenced homogeneous half-space (or a section thereof) under the effect of
gravity, which can be represented analytically in a stress point as follows:

\begin{equation}
    \sigma_{ini,vert}=\sigma_{ini,vert,h=0}+\gamma h
\end{equation}
\begin{equation}
    \sigma_{ini,horiz}=\sigma_{ini,vert} K_0 \quad \text{with } \quad K_0 = 1-\sin\varphi
\end{equation}

In these equations, $\gamma=\rho g$ corresponds to the weight of the soil, $h$
to the overburden height, $\sigma_{ini,vert,h=0}$ to the vertical stress at
$h=0$ and $K_0$ to the earth pressure coefficient. This earth pressure
coefficient is usually estimated as a function of the angle of internal
friction $\varphi$ (simplified equation according to Jaky).

\begin{lstlisting}[language=perl, caption={Definition of a geostatic stress state using ‘ComputeEigenstrainFromInitialStress’ },label={initial-conditions-ComputeEigenstrainFromInitialStress}]
[Functions]
    [ini_xx_yy]
      type = ParsedFunction
      vars = 'sig_top   y_top   rho      g    k0 '
      vals = '-1.5      0       0.0025   10   0.3'
      value = '(sig_top - rho * g * (y_top - y)) * k0'
    []
    [ini_zz]
      type = ParsedFunction
      vars = 'sig_top   y_top   rho      g'
      vals = '-1.5      0       0.0025   10'
      value = '(sig_top - rho * g * (y_top - y))'
    []
[]
  
[Materials]
    [strain]
      type = ComputeIncrementalSmallStrain
      volumetric_locking_correction=false
      eigenstrain_names = ini_stress
    []
    [ini_stress]
      type = ComputeEigenstrainFromInitialStress
      eigenstrain_name = ini_stress
      initial_stress = 'ini_xx_yy 0 0   0 ini_xx_yy 0   0 0 ini_zz'
    []
[]
\end{lstlisting}

\subsection{Initial stress state for anisotropic stresses}
\label{chap:IC-stress-state-anisotropic}

Align the model to the orientation of the initial stresses

(to be written)
