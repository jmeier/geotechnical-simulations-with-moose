\section{Initial conditions for Moose objects}
\label{chap:IC-moose-objects}

For definition of the initial (starting) conditions for the variables a Moose
simulation the \href{https://mooseframework.inl.gov/syntax/ICs}{ICs system} is
to be used.

(to be written)

\section{Initial stress state}
\label{chap:IC-stress-state}

The initial (starting) stress state of geotechnical simulations are often
non-trivial. The reason for this is that the initial state to be modelled is a
section of the geosphere with its often millions of years old history. In
addition, the ground surface is often non-horizontal, which is why the stress
trajectories near the surface are aligned accordingly. Furthermore, the initial
stress state is often anisotropic with a specific orientation.

\begin{itemize}
    \item Align the model to the orientation of the initial stresses
    \item Fix the model boundaries (usually all boundaries are only fixed in the
          direction of the normal and the lower boundary can also be fixed horizontally).
    \item Use an appropriate EigenstrainFromStress
\end{itemize}

(to be written)

