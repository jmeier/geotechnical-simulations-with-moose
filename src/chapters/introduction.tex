By enabling massively parallel multiphysics simulation the open-source, finite
element framework Moose
(\href{https://mooseframework.inl.gov}{mooseframework.inl.gov}) can also be
used for geotechnical design practice. This document aims to collect
information and techniques that are suitable for creating such geotechnical
simulations with Moose.

In the geotechnical planning practice, finite element models are often used for
deformation prognosis. Furthermore, such FE models can help to identify complex
failure mechanisms. The requirements for these models are accordingly a
sufficiently good representation of the geotechnical structure including the
subsoil, the groundwater conditions, pre-existing structures and similar. These
FE models simulate the construction process over time. As this design process
is generally intended to take place in a controlled and quasi-static form, the
associated simulations are usually also transient and quasi-static. Therefore,
scope and main focus of this document are transient quasi-static simulations.

This document is primarily intended as a collection of information rather than
a textbook or step-by-step guide. As such, it is intended to be a 'living'
document that will be continually edited and updated.

This document is published under the
\href{https://creativecommons.org/licenses/by-sa/4.0/}{CC-BY-SA-4.0} license.
No warranties are given.