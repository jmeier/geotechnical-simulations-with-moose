\section{General considerations on geometry}
\label{geometry-general}

When modelling geotechnical problems, it is often desirable to make a
meaningful geometric abstraction to avoid time-consuming modelling, but also to
avoid inaccuracies due to oversimplification. However, experience shows that
the effort involved in creating such a model geometry should not be
underestimated and that the models consist of a large number of individual
parts and components - often called ‘geometrical objects’. Some of these
objects have trivial shapes (e.g. rectangular surfaces for a sheet pile wall
segment), others can have non-trivial curvilinear boundaries (e.g. geological
bodies). All of these objects must be shaped to fit each other (without gaps or
overlaps) and together form the FE model. Usually, geotechnical models are
managed (e.g materials, staging, etc) on the level of these objects.

\section{‘Subdomains’ resp. ‘Blocks’}
\label{geometry-blocks-and-subdomains}

Geometrically, a Moose model consists of one or more ‘subdomains’. Due to the
use of \href{https://libmesh.github.io/}{libmesh} by Moose, the term ‘block’ is
also used as a synonym for ‘subdomain’ in many places. Element types and Moose
objects such as materials, kernels, etc. are assigned to these subdomains in
turn. Furthermore, each FE element belongs to exactly one subdomain.

It is therefore necessary to create at least one subdomain for each element
type and each material. To simplify the handling of construction stages, it has
also proved useful to further subdivide element sets into individual subdomains
so they match the ‘geometrical objects’ mentioned before so they can to be used
independently during the simulation (e.g. deactivated or activated at different
times). This document follows this approach that Moose models are managed on
the basis of subdomains/blocks.

As the shape of these individual parts and components directly influences the
shape of the FE elements they are build of, further prerequisites must be
considered during geometrisation: For instance, as few ‘unfavourably shaped’
(e.g. too acute-angled) FE elements as possible should result in order to avoid
numerical instabilities or load increments that are too small. A common reason
for acute-angled elements and large numbers of elements are corner or end
points that are close together and unfavourable intersections of the objects.
Another requirement for geometrisation is that the number of FE elements should
remain within a range that enables acceptable calculation times.

\section{Getting the mesh into Moose}
\label{geometry-getting-the-mesh-into-moose}

Although Moose's ‘Mesh System’ provides a variety of tools for generating FE
meshes on the basis of input file commands, the creation of geometrically
complex models with these commands is often confusing and error-prone.
% https://mooseframework.inl.gov/syntax/Mesh/index.html

Therefore, the mesh including all subdomains and FE-elements is prepared
outside of Moose and imported using the \codeword{FileMeshGenerator}. The
\href{https://gmsh.info/doc/texinfo/gmsh.html#MSH-file-format}{MSH file format}
has proven itself in this context. One possible workflow is to create the
geometry of the subdomains in \href{https://blender.org}{Blender}, then
generate the mesh using \href{https://gmsh.info/}{gmsdh} and save it as an MSH
file. See listing \autoref{geometry-FileMeshGenerator} for the code required in
a Moose input file for the import of an MSH file.
% \href{https://mooseframework.inl.gov/source/meshgenerators/FileMeshGenerator.html}{FileMeshGenerator}

\begin{lstlisting}[language=perl, float, caption={Read mesh from a file},label={geometry-FileMeshGenerator}]
[Mesh]
    [file]
        type = FileMeshGenerator
        file = "source.msh"
    []
    second_order = true
[]
\end{lstlisting}

\section{Lower dimensional blocks in MSH files}
\label{geometry-MSH}

A notable quirk of the MSH file format when used with Moose is the definition
of lower dimensional blocks. The term ‘lower dimensional blocks’ is used here
to describe subdomains with FE elements with a lower dimensionality (e.g.
shells and beams in a model with 3D solid elements). The MSH file format makes
no internal distinction as to whether blocks with lower dimensional elements
are to be interpreted as a side-set or as a ‘real’ element block. To introduce
this distinction, Moose recognises the codeword \codeword{lower_dimensional_block}
(although this codeword is not an official part of the MSH file
format). If this codeword is found in the line of an entry in the
\codeword{$PhysicalNames}, Moose assumes that this block does not represent a
sideset, but contains lower dimensional elements. As shown in
\autoref{geometry-MSH-LowerDimBlock} the codeword may be just appended at the
end of the line (without beeing part of the name of the physical name) or even
be part of the physical name itself.

\begin{lstlisting}[float, caption={Fragment of an MSH file containing ‘lower dimensional blocks’},label={geometry-MSH-LowerDimBlock}]
$MeshFormat
4.1 0 8
$EndMeshFormat
$PhysicalNames
(*@{\raisebox{-1pt}[0pt][0pt]{$\vdots$}}@*)
0 1 "PointA"                                         (*@{$\leftarrow$ side-set (node)}@*)
1 2 "topleftedge"                                    (*@{$\leftarrow$ side-set (line)}@*)
1 3 "lineSignPole000" lower_dimensional_block        (*@{$\leftarrow$ block with beam-elements}@*)
1 4 "lineSignPole001_lower_dimensional_block"        (*@{$\leftarrow$ block with beam-elements}@*)
2 5 "BoundaryZMin"                                   (*@{$\leftarrow$ side-set (surface)}@*)
2 6 "Sign" lower_dimensional_block                   (*@{$\leftarrow$ block with shell-elements}@*)
(*@{\raisebox{-1pt}[0pt][0pt]{$\vdots$}}@*)
\end{lstlisting}

\section{Choice of element types}
\label{geometry-element-types}

For obvious reasons, the element types of the individual subdomains must be
compatible with each other (e.g. the number and distribution of nodes must
match, etc). Given the often complex geometry of geotechnical models (see
\autoref{geometry-general}), elements with triangular faces are often
preferred. In this case, volume elements are correspondingly tetrahedral (TET)
and surface elements are triangular (TRI). However, since linear TET and TRI
elements (TET4 and TRI3) have very poor numerical properties (e.g.
unrealistically high stiffness, locking), ‘quadratic’ elements (TET10 and TRI6)
must be used. This leads to the commonly used compatible element types for
tetrahedral/triangular discretisation show in
\autoref{tab:geometry-TET-TRI-elements}.

\begin{table}
    \begin{tabularx}{\textwidth}{@{}lXl@{}}
        \hline
        Subdomain type
                             &
        Element type
                             &
        Comments
        \\

        \hline
        volume
                             &
        TET10
                             &
        tetrahedral element with 10 nodes and 4 integration points
        \\

        surface (e.g. shell) & TRI6  & triangular element with 6 nodes \\

        line (e.g. beam)     & EDGE3 & line element with 3 nodes       \\

        \hline
    \end{tabularx}
    \caption{Commonly used compatible element types for tetrahedral/triangular discretisation}
    \label{tab:geometry-TET-TRI-elements}
\end{table}

Currently, Moose has full support for TET10 elements, but not for EDGE3 and
TET6 elements. EDGE3 and TRI6 elements can currently be used in a Moose-app by
cloning the following Moose code-plugins as a ‘contrib’:

\begin{description}[font=$\bullet$~\normalfont]
    \item [beams:] \href{https://github.com/Kavan-Khaledi/moose-codeplugin-beam}{github.com/Kavan-Khaledi/moose-codeplugin-beam}
    \item [shells:] \href{https://github.com/Kavan-Khaledi/moose-codeplugin-shell}{github.com/Kavan-Khaledi/moose-codeplugin-shell}
\end{description}

\section{Groups of geometrical objects}
\label{geometry-groups}

Moose does not explicitly support groups geometrical objects but allows to
create variables in the input file holding names of several objects on one hand
and to use these variables directly for block-restriction (e.g. assignment of
materials).

Due to the fact that Moose throws an error if unused variables are found in the
input file, but the user may want to define groups even if they are not
(currently) used, it has proven useful to add 'fake users' using a
ParsedFunction as shown in \autoref{geometry-moose-variable-groups}.

% HTML VRML bash hansl Python GAP Awk make Gnuplot Perl PHP SPARQL ksh
\begin{lstlisting}[language=Perl, float, caption={Moose input file fragment: variable holding several block names and fake-user},label={geometry-moose-variable-groups}]
[Mesh]
    # Variable holding names of all blocks with volumes
    Volumes = 'Base Column Suzanne'
[]

[Functions]
    # Fake user for the variable above
    [FakeUser_Volumes]
        type = ParsedFunction
        expression = 'a'
        symbol_names = 'a'
        symbol_values = '1'
        control_tags = ${Mesh/Volumes}
    []
[]
\end{lstlisting}

