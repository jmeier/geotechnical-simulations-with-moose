\section{General Considerations}
\label{geometry-general}

When modelling geotechnical problems, it is often desirable to
make a meaningful geometric abstraction to avoid time-consuming
modelling, but also to avoid inaccuracies due to oversimplification.
However, experience shows that the effort involved in creating such
a model geometry should not be underestimated and that the models
consist of a large number of individual parts and components.
Some of these objects have trivial shapes (e.g. rectangular surfaces
for a sheet pile wall segment), others can have non-trivial curvilinear
boundaries (e.g. geological bodies).
All of these objects must be shaped to fit each other (without gaps or
overlaps) and together form the FE model.

As the shape of these individual parts and components directly influences
the shape of the FE elements they are build of, further prerequisites must
be considered during geometrisation: For instance, as few ‘unfavourably
shaped’ (e.g. too acute-angled) FE elements as possible should result in
order to avoid numerical instabilities or load increments that are too
small. A common reason for acute-angled elements and large numbers of
elements are corner or end points that are close together and unfavourable
intersections of the objects. Another requirement for geometrisation
is that the number of FE elements should remain within a range that
enables acceptable calculation times.

Although Moose's ‘Mesh System’ provides a variety of tools for generating
FE meshes on the basis of input file commands, the creation of geometrically
complex models with these commands is often confusing and error-prone.
% https://mooseframework.inl.gov/syntax/Mesh/index.html

The mesh is therefore often created outside of the Mosse (e.g. using
\href{https://blender.org}{Blender}), saved as MSH-file and then imported
using the \codeword{FileMeshGenerator}.
The code required for this import in a Moose input file is shown in
listing \ref{FileMeshGenerator}.
% \href{https://mooseframework.inl.gov/source/meshgenerators/FileMeshGenerator.html}{FileMeshGenerator}

\begin{lstlisting}[caption={Read mesh from a file},label={FileMeshGenerator}]
[Mesh]
    [file]
        type = FileMeshGenerator
        file = "source.msh"
    []
    second_order = true
[]
\end{lstlisting}

(to be written)
