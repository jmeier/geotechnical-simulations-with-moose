\section{General Considerations}
\label{geometry-general}

When modelling geotechnical problems, it is often desirable to make a
meaningful geometric abstraction to avoid time-consuming modelling, but also to
avoid inaccuracies due to oversimplification. However, experience shows that
the effort involved in creating such a model geometry should not be
underestimated and that the models consist of a large number of individual
parts and components - often called ‘geometrical objects’. Some of these
objects have trivial shapes (e.g. rectangular surfaces for a sheet pile wall
segment), others can have non-trivial curvilinear boundaries (e.g. geological
bodies). All of these objects must be shaped to fit each other (without gaps or
overlaps) and together form the FE model. Usually, geotechnical models are
managed (e.g materials, staging, etc) on the level of these objects.

\section{‘Subdomains’ resp. ‘Blocks’}
\label{geometry-blocks-and-subdomains}

Geometrically, a Moose model consists of one or more ‘subdomains’. Due to the
use of \href{https://libmesh.github.io/}{libmesh} by Moose, the term ‘block’ is
also used as a synonym for ‘subdomain’ in many places. Element types and Moose
objects such as materials, kernels, etc. are assigned to these subdomains in
turn. Furthermore, each FE element belongs to exactly one subdomain.

It is therefore necessary to create at least one subdomain for each element
type and each material. To simplify the handling of construction stages, it has
also proved useful to further subdivide element sets into individual subdomains
so they match the ‘geometrical objects’ mentioned before so they can to be used
independently during the simulation (e.g. deactivated or activated at different
times). This document follows this approach that Moose models are managed on
the basis of subdomains/blocks.

As the shape of these individual parts and components directly influences the
shape of the FE elements they are build of, further prerequisites must be
considered during geometrisation: For instance, as few ‘unfavourably shaped’
(e.g. too acute-angled) FE elements as possible should result in order to avoid
numerical instabilities or load increments that are too small. A common reason
for acute-angled elements and large numbers of elements are corner or end
points that are close together and unfavourable intersections of the objects.
Another requirement for geometrisation is that the number of FE elements should
remain within a range that enables acceptable calculation times.

\section{Getting the mesh into Moose}
\label{geometry-getting-the-mesh-into-moose}

Although Moose's ‘Mesh System’ provides a variety of tools for generating FE
meshes on the basis of input file commands, the creation of geometrically
complex models with these commands is often confusing and error-prone.
% https://mooseframework.inl.gov/syntax/Mesh/index.html

Therefore, the mesh including all subdomains and FE-elements is prepared
outside of Moose and imported using the \codeword{FileMeshGenerator}. The
\href{https://gmsh.info/doc/texinfo/gmsh.html#MSH-file-format}{MSH file format}
has proven itself in this context. One possible workflow is to create the
geometry of the subdomains in \href{https://blender.org}{Blender}, then
generate the mesh using \href{https://gmsh.info/}{gmsdh} and save it as an MSH
file. See listing \ref{FileMeshGenerator} for the code required in a Moose
input file for the import of an MSH file.
% \href{https://mooseframework.inl.gov/source/meshgenerators/FileMeshGenerator.html}{FileMeshGenerator}

\begin{lstlisting}[caption={Read mesh from a file},label={FileMeshGenerator}]
[Mesh]
    [file]
        type = FileMeshGenerator
        file = "source.msh"
    []
    second_order = true
[]
\end{lstlisting}

\todoinline{LowerDimensionalBlocks in MSH-files }

