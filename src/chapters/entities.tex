\section{Overview}
\label{chap:entities-overview}

This chapter collects information on how different types of geometric entities,
structures, loads etc. can be modelled. \autoref{tab:entities-overview}
contains a list of common geotechnical components and associated options that
can be used for modelling them.

\begin{table}
  \begin{tabularx}{\textwidth}{@{}lXl@{}}
    \hline
    Entity
     &
    Options for Modelling
     &
    Reference
    \\

    \hline
    soil volume
     &
    \bulleted{cluster of volume elements}
     &
    \shortautoref{chap:entities-volume}
    \\

    \hline
    excavation pit wall
     &
    \bulleted{shell elements}
     &
    \shortautoref{chap:entities-shell}
    \\

     &
    \bulleted{cluster of volume elements in case of a 'thick' wall (e.g. slurry wall, bored pile wall)}
     &
    \shortautoref{chap:entities-volume}
    \\

    \hline
    concrete volume

     &
    \bulleted{cluster of volume elements}
     &
    \shortautoref{chap:entities-volume}
    \\

     &
    \bulleted{shell elements in case of a 'thin' walls}
     &
    \shortautoref{chap:entities-shell}
    \\

    \hline
    sprayed concrete
     &
    \bulleted{shell elements}
     &
    \shortautoref{chap:entities-shell}
    \\

    \hline
    pile, soil nail
     &
    \bulleted{cluster of volume elements in case of a pile with large diameter (e.g. bored pile)}
     &
    \shortautoref{chap:entities-volume}
    \\

    %  &
    % \bulleted{embedded beams}
    %  &
    % \shortautoref{chap:entities-embedded-beams}
    % \\

    \hline
    prop
     &
    \bulleted{spring between nodes}
     &
    \shortautoref{chap:entities-springs}
    \\

     &
    \bulleted{fixed end spring}
     &
    \shortautoref{chap:entities-fixed-end-springs}
    \\

     &
    \bulleted{beam}
     &
    \shortautoref{chap:entities-beams}
    \\

    \hline
    ground anchor
     &
    \bulleted{spring between nodes + (embedded) beam}
     &
    \shortautoref{chap:entities-springs}
    \\

    \hline
  \end{tabularx}
  \caption{Selected geotechnical entities and their respective modelling}
  \label{tab:entities-overview}
\end{table}

\section{Entities}
% \label{chap:entities}

\subsection{Volumes}
\label{chap:entities-volume}

To model a volume, such as a soil volume, concrete volume, etc., the following
is required:

\begin{description}[font=$\bullet$~\normalfont]
  \item [subdomain:] A subdomain of FE-elements of an appropriate element type (e.g. TET10).
  \item [materials:] Assignment of Moose materials defining the constitutive behaviour.
\end{description}

\subsection{FixedEndSprings}
\label{chap:entities-fixed-end-springs}

Fixed end springs are currently not directly supported by Moose. As a
workaround one can define a spring between nodes
(\autoref{chap:entities-springs}) and fix the other node.

\href{https://mooseframework.inl.gov/source/nodalkernels/CoupledForceNodalKernel.html}{CoupledForceNodalKernel}

\href{https://onlinelibrary.wiley.com/doi/pdf/10.1002/pamm.202200045}{pamm.202200045}

\subsection{Springs between nodes (NodeToNodeAnchors)}
\label{chap:entities-springs}

% Springs between nodes are directly supported by Moose. 
(to be written)

mastodon:
\href{https://mooseframework.inl.gov/mastodon/source/materials/LinearSpring.html}{LinearSpring}

\subsection{Beams (line element)}
\label{chap:entities-beams}

To model beams, the following is required:

\begin{description}[font=$\bullet$~\normalfont]
  \item [subdomain:] A subdomain of FE-elements of an appropriate element type (e.g. EDGE3).
  \item [materials:] Assignment of Moose materials defining the constitutive behaviour of the beam.
\end{description}

Currently, Moose has no support for appropriate beam elements and materials for
EDGE3. In your MooseApp, the beam material for EDGE may be included as an
‘contrib’ by cloning into
\href{https://github.com/Kavan-Khaledi/moose-codeplugin-beam}{github.com/Kavan-Khaledi/moose-codeplugin-beam}

% \subsection{EmbeddedBeams (line element)}
% \label{chap:entities-embedded-beams}

\subsection{Shell (surface element)}
\label{chap:entities-shell}

To model shells (or "plates"), the following is required:

\begin{description}[font=$\bullet$~\normalfont]
  \item [subdomain:] A subdomain of FE-elements of an appropriate element type (e.g. TRI6).
  \item [materials:] Assignment of Moose materials defining the constitutive behaviour of the shell.
\end{description}

Currently, Moose has no support for appropriate shell elements and materials
for TRI6. In your MooseApp, the shell material for TRI6 may be included as an
‘contrib’ by cloning into
\href{https://github.com/Kavan-Khaledi/moose-codeplugin-shell}{github.com/Kavan-Khaledi/moose-codeplugin-shell}

\subsection{Interfaces (surface element)}

(to be written)

\subsection{Contact}

(to be written)

\subsection{Pore Pressure Boundaries}

(to be written)

\section{Groundwater}
\label{chap:entities-groundwater}

Groundwater can affect subsoil and geotechnical structures in many ways. Below
the water table it is generally assumed that the pore space is completely
saturated with water. In this fully saturated zone, the pore water pressure
caused by gravity acts on the grain structure in all directions, creating
buoyancy and thereby influencing the contact stresses between the grains. If
the pore water is in motion, for instance as a consequence of an external
pressure gradient, flow forces will act upon the grain structure. A variety of
additional effects can be triggered by temperature changes (e.g. freezing),
quick changes in the external loading including cyclic or dynamic loading, and
many more.

In partially saturated zones only a part of the pore space is filled with water
and the rest with gas (usually air). In geotechnics, the effects of the gas is
often neglected. This region has its own set of effects, including an often
reduced permeability in relation to the fully saturated zone. The reduced
permeability can be explained by the fact that the surface tension causes the
water to retreat into the corners of the grain structure, leaving fewer direct
connections between individual water droplets and thus fewer pathways for mass
transport.

The complete absence of the liquid phase (e.g. water) is referred to as a dry
state. Again, in geotechnics the effects of the gas is often neglected.

For partially and fully saturated conditions, a fraction of the external loads
on a body of soil is transferred via the grain to grain contacts and the rest
via the pore pressure. This observation leads to the concept of ‘effective
stresses’ (e.g. \cite{doi:10.1680/geot.1962.12.2.125},
\cite{doi:10.1680/sposm.02050.0014}) where the total stress
$\sigma_\mathrm{tot}$ (often denoted also just by $\sigma$) equals the sum of
the effective stress $\sigma_\mathrm{eff}$ (often denoted also just by
$\sigma'$) and the pore water pressure $p_\mathrm{w}$:

\begin{equation}
  \sigma_\mathrm{tot} = \sigma_\mathrm{eff} + p_\mathrm{w}
\end{equation}
or, using the shorter notation:
\begin{equation}
  \sigma = \sigma' + p_\mathrm{w}
\end{equation}

(to be written)

\begin{table}
  \begin{tabularx}{\textwidth}{@{}llYY@{}}
    \hline
     &
     &
    \multicolumn{2}{c}{text}
    \\

     &
    ?
     &
    A
     &
    B
    \\

    \hline
    \multirow{2}{*}{text}
     &
    ?
     &
    stationary
     &
    transient
    \\

     &
    ?
     &
    not coupled
     &
    coupled
    \\

    \hline
  \end{tabularx}
  \caption{???}
  \label{tab:entities-water-simulation-types}
\end{table}
