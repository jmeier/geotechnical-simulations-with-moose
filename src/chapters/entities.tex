\section{Overview}
\label{chap:entities-overview}

This chapter collects information on how different types of geometric entities,
structures, loads etc. can be modelled. \autoref{tab:entities-overview}
contains a list of common geotechnical components and associated options that
can be used for modelling them.

\begin{table}
    \begin{tabularx}{\textwidth}{@{}lXl@{}}
        \hline
        Entity
         &
        Options for Modelling
         &
        Reference
        \\

        \hline
        soil volume
         &
        \bulleted{cluster of volume elements}
         &
        \shortautoref{chap:entities-volume}
        \\

        \hline
        excavation pit wall
         &
        \bulleted{shell elements}
         &
        \shortautoref{chap:entities-shell}
        \\

         &
        \bulleted{cluster of volume elements in case of a 'thick' wall (e.g. slurry wall, bored pile wall)}
         &
        \shortautoref{chap:entities-volume}
        \\

        \hline
        concrete volume

         &
        \bulleted{cluster of volume elements}
         &
        \shortautoref{chap:entities-volume}
        \\

         &
        \bulleted{shell elements in case of a 'thin' walls}
         &
        \shortautoref{chap:entities-shell}
        \\

        \hline
        sprayed concrete
         &
        \bulleted{shell elements}
         &
        \shortautoref{chap:entities-shell}
        \\

        \hline
        pile, soil nail
         &
        \bulleted{cluster of volume elements in case of a pile with large diameter (e.g. bored pile)}
         &
        \shortautoref{chap:entities-volume}
        \\

        %  &
        % \bulleted{embedded beams}
        %  &
        % \shortautoref{chap:entities-embedded-beams}
        % \\

        \hline
        prop
         &
        \bulleted{spring between nodes}
         &
        \shortautoref{chap:entities-springs}
        \\

         &
        \bulleted{fixed end spring}
         &
        \shortautoref{chap:entities-fixed-end-springs}
        \\

         &
        \bulleted{beam}

         &
        \shortautoref{chap:entities-beams}
        \\

        \hline
        ground anchor
         &
        \bulleted{spring between nodes + (embedded) beam}
         &
        \shortautoref{chap:entities-springs}
        \\

        \hline

    \end{tabularx}
    \caption{Selected geotechnical entities and their respective modelling}
    \label{tab:entities-overview}
\end{table}

\section{Entities}
% \label{chap:entities}

\subsection{Volumes}
\label{chap:entities-volume}

\subsection{FixedEndSprings}
\label{chap:entities-fixed-end-springs}

https://mooseframework.inl.gov/source/nodalkernels/CoupledForceNodalKernel.html

https://onlinelibrary.wiley.com/doi/pdf/10.1002/pamm.202200045

\subsection{Springs between nodes (NodeToNodeAnchors)}
\label{chap:entities-springs}

https://mooseframework.inl.gov/mastodon/source/materials/LinearSpring.html

\subsection{Beams (line element)}
\label{chap:entities-beams}

% \subsection{EmbeddedBeams (line element)}
% \label{chap:entities-embedded-beams}

\subsection{Shell (surface element)}
\label{chap:entities-shell}

\subsection{Interfaces (surface element)}

\subsection{Contact}

\subsection{Pore Pressure Boundaries}
