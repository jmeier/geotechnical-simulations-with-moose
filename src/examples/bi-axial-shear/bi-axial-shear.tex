\section{Problem statement}

A bi-axial shear test of a homogenous material block of $\SI{1}{\metre} \times
    \SI{1}{\metre} \times \SI{1}{\metre}$ should be modeled to test the limit load
resulting from the Mohr-Coulomb failure criterion.

\begin{table}[htbp]
    \centering
    \caption{Material parameters}
    \label{bi-axial-shear-mohr-coulomb:material-parameters}
    \begin{tabularx}{\textwidth}{XYY}

        \hline

        Property                        & Physical unit                                         & Value       \\

        \hline

        Youngs modulus $E$              & \si[per-mode = symbol]{\kilo\newton\per\square\metre} &
        \SI{1000}{}                                                                                           \\

        Poisson's ratio $\nu$           & -                                                     & \SI{0.25}{} \\

        Angle of inner friction $\phi'$ & \si[per-mode = symbol]{\degree}                       & \SI{30}{}
        \\

        Cohesion $c'$                   & \si[per-mode = symbol]{\kilo\newton\per\square\metre} & 1           \\

        \hline
    \end{tabularx}
\end{table}

\section{Analytical solution}

From the yield function of the Mohr-Coulomb failure criterion shown in
\autoref{eqn:Mohr-Coulomb-failure-criterion} the limit load of the bi-axial
test can be derived as given in \autoref{eqn:bi-axial-test-limit-load}.

\begin{equation}
    \label{eqn:Mohr-Coulomb-failure-criterion}
    f = \frac{\abs{\sigma'_1 - \sigma'_2}}{2} + \frac{\sigma'_1 + \sigma'_2}{2} \cdot \sin{\phi'} - c' \cdot \cos{\phi'} = 0
\end{equation}

\begin{equation}
    \label{eqn:bi-axial-test-limit-load}
    \sigma'_1 = \sigma'_2 \cdot \frac{1 + \sin{\phi'}}{1 - \sin{\phi'}} - 2c' \cdot \frac{\cos{\phi'}}{1 - \sin{\phi'}}
    \approx \qty{-6.464}{\kilo\newton\per\square\metre}
\end{equation}

\section{Moose}

In the corresponding Moose model of this quasi-static problem, a transient
executioner is used. The load $\sigma'_1$ linearly with time so that at $t =
    \qty{6.464}{\second}$ the load is $\sigma'_1 =
    \qty{-6.464}{\kilo\newton\per\square\metre}$. Due to the ideal-plastic nature
of the material behaviour, Moose will attempt to calculate time steps and
iteratively reduce the time step if convergence is not achieved. Eventually the
last converged time step should be at $t \approx \qty{6.464}{\second}$.

This model was calculated using the ‘PJFNK’ solver. The time step is initially
chosen to be \qty{0.25}{\second} and the minimum time step is limited to
\qty{0.001}{\second}. Due to the automatic cut-back of the time steps close to
failure, the last converged time step is at $t=\qty{6.43945}{\second}$. The
Moose input file for this model is attached to this document by the name of
‘bi-axial-shear.i’.

\fileattachment{\currfiledir/bi-axial-shear.i}{bi-axial-shear.i}

\section{Plaxis 3D}
