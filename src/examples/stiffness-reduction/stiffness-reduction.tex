\section{Problem statement}

The stiffness of materials can change over time as a result of certain
processes. One example is the temperature rise of a rubber block, which becomes
‘softer’ and ‘slumps down’ as a result of this temperature load. Similar
effects may be observed as a result of weathering of geomaterials.

To simulate this effect in a simplified way, the reduction of stiffness of a
block of $\SI{1}{\metre} \times \SI{1}{\metre} \times \SI{1}{\metre}$ is to be
simulated. While the (vertical) sides are fixed horizontally and the bottom is
fixed vertically, the top surface loaded with a pressure of $\sigma_{top} =
    \SI[per-mode = symbol]{1.0}{\kilo\newton\per\square\metre}$. Two variants are
to be simulated, the first without dead load and the second with the block also
loaded by gravity. The soil material is assumed to be purely linear-elastic and
no water present. Material parameters are given in
\autoref{stiffness-reduction:material-parameters}. The stiffness of the
material was chosen with an arbitrary initial value of $E_0 = \SI[per-mode =
        symbol]{1000}{\kilo\newton\per\square\metre}$ to be reduced to $E_1 =
    \SI[per-mode = symbol]{500}{\kilo\newton\per\square\metre}$.

Vertical displacement $u_z$ at the top of the block is to be determined as
simulation result.

\begin{table}[htbp]
    \centering
    \caption{Material parameters}
    \label{stiffness-reduction:material-parameters}
    \begin{tabularx}{\textwidth}{XYY}

        \hline

        Property                                     & Physical unit                & Value      \\

        \hline

        initial Youngs modulus $E_0$                 & \si[per-mode =
        symbol]{\kilo\newton\per\square\metre}       & \SI{1000}{}                               \\

        final Youngs modulus $E_1$                   & \si[per-mode =
        symbol]{\kilo\newton\per\square\metre}       & \SI{500}{}                                \\

        Poisson's ratio $\nu$                        & -                            & \SI{0.0}{} \\

        density (weightless $\vert$ finite) $\rho_s$ & \si[per-mode =
        symbol]{\kilogram\per\cubic\metre}           & \SI{0}{} $\vert$ \SI{2500}{}              \\

        \hline
    \end{tabularx}
\end{table}

\section{Analytical solution}

In the case of $\nu = 0.0$ given here, the vertical deformation $\Delta h$ can
be analytical determined from Hooke's law
(\autoref{stiffness-reduction:stiffness-modulus}):

\begin{equation}
    \label{stiffness-reduction:stiffness-modulus}
    E = \frac{\sigma_{zz}}{\varepsilon_{zz}} \rightarrow \varepsilon_{zz}
    = \frac{\sigma_{zz}}{E} = \frac{\Delta h}{h}
\end{equation}

For the weightless material this leads to:

\begin{equation}
    u_z = \frac{\sigma_{top}}{E_1} - \frac{\sigma_{top}}{E_0} = \SI{0.001}{\metre} = \SI{1}{\milli\metre}
\end{equation}

For the material with $\rho_s = \qty[per-mode
        =symbol]{2500}{\kilogram\per\cubic\metre}$ also subjected to a gravitational
acceleration of $g = \qty[per-mode = symbol]{9.81}{\metre\per\square\second}$,
the settlement of the upper boundary can be calculated as for a spring under
its own weight:

\begin{equation}
    u_z = \left[ \frac{\sigma_{top}}{E_1} - \frac{\sigma_{top}}{E_0} \right] + \left[\frac{\rho_s \cdot g \cdot h}{2 \cdot E_1} - \frac{\rho_s \cdot g \cdot h}{2 \cdot E_0} \right] = \qty{1}{\milli\metre} + \qty{12.26}{\milli\metre} = \qty{13.26}{\milli\metre}
\end{equation}

\section{Numerical solution with Moose}

The system of physical units for the Moose model was set to metre
(\unit{\metre}), seconds (\unit{\second}), and tons (\unit{\ton}). Due to this,
the physical unit for stress and pore-pressures is \unit[per-mode =
    symbol]{\kilo\newton\per\square\metre}.

The mesh used for Moose consists of \qty{8245}{} nodes forming \qty{5184}{}
TET10 elements leading to \qty{24735}{\DOF}. Due to its pure linear-elastic
setup, this model was calculated using the ‘NEWTON’ solver.

The Moose input file for this model is attached to this document by the name of
‘stiffness-reduction.i’.

\fileattachment{\currfiledir/stiffness-reduction.i}{stiffness-reduction.i}

\begin{table}[htbp]
    \centering
    \caption{Selected results of the Moose models}
    \label{fully-saturated-undrained-compression:moose-results}

    \begin{tabularx}{0.972\hsize}{
            >{\hsize=0.6\hsize\linewidth=\hsize}X
            >{\hsize=0.4\hsize\linewidth=\hsize}X}
        {\begin{tabularx}{1\linewidth}{
                     >{\hsize=0.4\hsize\linewidth=\hsize}X
                     >{\hsize=0.25\hsize\linewidth=\hsize\hspace{-5pt}}Y
                     >{\hsize=0.25\hsize\linewidth=\hsize\hspace{-5pt}}Y}
                 \toprule
                 Model response                                                                                & \multicolumn{2}{c}{Material density}                                                                                   \\
                 {}                                                                                            & \qty[per-mode = symbol]{0}{\kilo\newton\per\cubic\metre} & \qty[per-mode = symbol]{2500}{\kilo\newton\per\cubic\metre} \\
                 \midrule

                 \multicolumn{3}{l}{Initial Conditions:}                                                                                                                                                                                \\
                 \hspace{1em} $\sigma'_{zz,z_{max}}$ (\unit[per-mode = symbol]{\kilo\newton\per\square\metre}) & \qty{\pm0.0}{}                                           & \qty{-1.0}{}                                                \\
                 \hspace{1em} $\sigma'_{zz,z_{min}}$ (\unit[per-mode = symbol]{\kilo\newton\per\square\metre}) & \qty{\pm0.0}{}                                           & \qty{-25.53}{}                                              \\

                 {}                                                                                            & {}                                                       & {}                                                          \\

                 \multicolumn{3}{l}{Boundary at $z_{max}$ after stiffness reduction:}                                                                                                                                                   \\
                 \hspace{1em} $u_z$            (\unit[per-mode = symbol]{\metre})                              & \qty{-1.034E-3}{}                                        & \qty{-13.41E-3}{}                                           \\
                 \bottomrule
             \end{tabularx}}
         &
        {\raisebox{0.0mm}
                {\begin{tikzpicture}[baseline=(current bounding box.center)]
                         \node[inner sep=0pt] (ch-stresses) at (0.35,0.0) {\includegraphics[width=0.925\linewidth]{\currfiledir stiffness-reduction.i.png}};
                         \node[draw,circle,fill=white,minimum size=.95cm,inner sep=0pt] at (+3.3,+2.2) {$z_{max}$};
                         \node[draw,circle,fill=white,minimum size=.95cm,inner sep=0pt] at (+3.1,-1.6) {$z_{min}$};
                     \end{tikzpicture}}
            }
        \\
    \end{tabularx}
\end{table}