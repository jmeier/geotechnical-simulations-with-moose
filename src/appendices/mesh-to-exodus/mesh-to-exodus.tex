For debugging purposes, it is sometimes necessary to write the mesh of a Moose
model to an ExodusII file without using other functions of Moose. This appendix
shows two ways to do this.

\section{Command line option: '--mesh-only'}

By adding the command line option \codeword{--mesh-only} when executing a Moose
input file via the command line, Moose will just output an ExodusII file for
the user to inspect. Please note, this will write an file ending with '*\_in.e'
\emph{to the app folder} (not the folder the input file ending with '.i' file
is located).

More information may be found in the link below in subsection 'Outputting The
Mesh':

\href{https://mooseframework.inl.gov/syntax/Mesh/#187a6146-3c09-448f-af5c-6c1230daa1af}{mooseframework.inl.gov/syntax/Mesh}

\section{Minimal Moose Input File}

\autoref{mesh-to-exodus:lst} shows a minimal input file reading a msh file
using a \codeword{FileMeshGenerator} and writing the model to an exodus file.

\begin{lstlisting}[language=Moose, caption={Read mesh from a file and export to ExodusII},label={mesh-to-exodus:lst}]
[Mesh]
    [file]
        type = FileMeshGenerator
        file = "FILENAME.msh"
    []
[]

[Problem]
    solve = false
[]

[Executioner]
    type = Steady
[]

[AuxVariables]
    [dummy]
        order = FIRST
    []
[]

[Outputs]
    exodus = true
[]
\end{lstlisting}