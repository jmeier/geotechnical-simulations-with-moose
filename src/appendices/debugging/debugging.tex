Moose is a wonderful platform for writing your own kernels, materials and more.
And as always when writing code, it must be debugged. This appendix collects
information on different techniques supporting code debugging.

As a good starting point, the Moose documentation has a page dedicated to
debugging. It is recommended to consider this page first:

\href{https://mooseframework.inl.gov/application_development/debugging.html}{mooseframework.inl.gov/application\_development/debugging.html}

In any case, it is recommended to compile the Moose app in the debug module for
troubleshooting and debugging. This can be achieved by using the command shown
in \autoref{app-debugging-METHOD=dbg} to compile. In
\autoref{app-debugging-METHOD=dbg} the number 8 should be replaced by the
number of processes the compile process should use.

\begin{lstlisting}[language=bash, caption={Compile in dbg-mode},label={app-debugging-METHOD=dbg}]
METHOD=dbg make -j 8
\end{lstlisting}

After compiling, the dbg-variant may be executed as shown in
\autoref{app-debugging-running-dbg} (again, replace the number 16 with the
number of processes the run should use). Please note the suffix \codeword{-dbg}
denoting the debug-variant. For details on this command line please see
\autoref{chap:patterns-running-moose}.

\begin{lstlisting}[language=bash, caption={Executing the dbg-variant of a moose-app},label={app-debugging-running-dbg}]
time mpiexec -n 16 ./your-app-name-dbg -i ./path-to-input-file.i
\end{lstlisting}

\importantbox{
    It is strongly recommended to use a debug exeutable for debugging. Although
    this form of executable can require significantly more wall time for the
    calculation, it contains additional information that is extremely helpful
    for the debugger and the programmer.
}

\section{Debugging in VSCode}

Up to the knowledge of the author, there seem to be no good solution on how to
use the build-in debug capabilities of VSCode for Moose-Apps. Please see the
link below.

\href{https://github.com/idaholab/moose/discussions/28328}{github.com/idaholab/moose/discussions/28328}

\section{Writing to ‘std::cout’ (you should avoid that)}

This is an option you should avoid. One may use the \codeword{std::cout}
function to print status data from your C++ code to the terminal output. The
disadvantages of this approach include the fact that the Moose terminal output
may be greatly inflated (and the information sought may not be found for this
very reason) and that the regular source code is mixed with source code for
debugging. Additionally, the Moose community does not accept code with this
call. As an alternative, please consider debugging with the help of a debugger
(e.g. see \autoref{app-debugging-gdb}).

\begin{lstlisting}[language=C++, caption={Using ‘std::cout’ to print to the terminal (standard output)},label={app-debugging-cout}]
std::cout << "CartesianLocalCoordinateSystem: '" << name() << "'\n";
\end{lstlisting}

\section{Using ‘gdb’}
\label{app-debugging-gdb}

On Linux, the debugger ‘gdb’ may be used to debug Moose-Apps on the command
line. \autoref{app-debugging-gdb-execute} shows the command to be used to
execute a Moose model with gdb.

\begin{lstlisting}[language=bash, caption={Run gdb with a Moose model},label={app-debugging-gdb-execute}]
gdb --args ./your-app-name-dbg -i models/path-to-input-file.i
\end{lstlisting}

After execution of ‘gdb’, several commands and options may be issued before
running the Moose-App itself. \autoref{app-debugging-gdb-example} shows an
example run.

\begin{lstlisting}[language=bash, caption={Example run of gdb},label={app-debugging-gdb-example}]
break libmesh_handleFPE     # Add breakpoint for 'libmesh_handleFPE'
catch throw                 # Add breakpoint for thrown errors
run                         # Run the program until a breakpoint or an error
c                           # Continue the program until a breakpoint or an error
bt                          # Print the stack trace
p var                       # Print a Variable
\end{lstlisting}

To raise breakpoints from within your code, the technique shown in
\autoref{app-debugging-breakpoint} may be used.

\begin{lstlisting}[language=C++, caption={C++ code to create a breakpoint for debugging},label={app-debugging-breakpoint}]
raise(SIGTRAP);             // this may need #include <signal.h>
\end{lstlisting}

\begin{table}[htbp]
    \centering
    \caption{Selection of commands of ‘gdb’}
    \label{app-debugging-tab:gdb}
    \begin{tabularx}{\textwidth}{
            >{\hsize=0.20\hsize\linewidth=\hsize}X
            >{\hsize=0.80\hsize\linewidth=\hsize}X}
        \hline
        command          & description                                   \\

        \hline

        b main           &

        Puts a breakpoint at the beginning of the program                \\

        b                &

        Puts a breakpoint at the current line                            \\

        b N              &

        Puts a breakpoint at line N                                      \\

        b +N             &

        Puts a breakpoint N lines down from the current line             \\

        b fn             &

        Puts a breakpoint at the beginning of function "fn"              \\

        d N              &

        Deletes breakpoint number N                                      \\

        info break       &

        list breakpoints                                                 \\

        info var [regex] &

        show names, types of global variables (all, or matching regex)   \\

        info variables   &

        list "All global and static variable names" (huge list)          \\

        info locals      &

        list "Local variables of current stack frame" (names and values), including
        static variables in that function.                               \\

        info args        &

        list "Arguments of the current stack frame" (names and values).  \\

        r                &

        Runs the program until a breakpoint or error                     \\

        c                &

        Continues running the program until the next breakpoint or error \\

        f                &

        Runs until the current function is finished                      \\

        s                &

        Runs the next line of the program                                \\

        s N              &

        Runs the next N lines of the program                             \\

        n                &

        Like s, but it does not step into functions                      \\

        u N              &

        Runs until you get N lines in front of the current line          \\

        p var            &

        Prints the current value of the variable "var"                   \\

        bt               &

        Prints a stack trace                                             \\

        up               &

        Goes up a level in the stack                                     \\

        down             &

        Goes down a level in the stack                                   \\

        q                &

        Quits gdb                                                        \\

        \hline
    \end{tabularx}
\end{table}