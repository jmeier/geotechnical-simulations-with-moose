\documentclass[12pt,a4paper]{report}
    % General document formatting
    \usepackage[margin=0.7in]{geometry}
    \usepackage[parfill]{parskip}
    \usepackage[utf8]{inputenc}
    \usepackage[]{mdframed}
    \usepackage{tikz}
    \usetikzlibrary{shapes.geometric, arrows, arrows.meta, chains, positioning}
    \usepackage{graphicx}
    \usepackage[edges]{forest}
    \usetikzlibrary{decorations.pathreplacing}
    
    % Related to math
    \usepackage{amsmath,amssymb,amsfonts,amsthm}

    % make links clickable
    \usepackage{hyperref}
    \hypersetup{
        colorlinks    = true,    % Colours links instead of ugly boxes
        urlcolor      = blue,    % Colour for external hyperlinks
        linkcolor     = black,   % Colour of internal links
        bookmarksopen = true
        %citecolor     = red      % Colour of citations
    }
    \usepackage[nameinlink,noabbrev]{cleveref}

    % Other packages
    \usepackage[inline]{enumitem}
    \usepackage{soul}   %for highlighting text
    % \usepackage{layouts}
    \usepackage{fontawesome5}

    \usepackage{filemod}
    
    \tikzstyle{startstop} = [rectangle, rounded corners, minimum width=3cm,
    minimum height=1cm, text centered, draw=black]
    \tikzstyle{process} = [rectangle, minimum width=3cm,
    minimum height=1cm, text centered, draw=black]
    \tikzstyle{decision} = [diamond, minimum width=3cm,
    minimum height=1cm, text centered, draw=black]
    \tikzstyle{arrow} = [thick, ->, >=stealth]


\title{Geotechnical Simulations with Moose}

\author{Jörg Meier}

\date{08.10.2024}

\begin{document}

%\maketitle
\begin{titlepage}
    \begin{flushright}
        \includegraphics[width=4.5cm]{img/gruner.pdf}
    \end{flushright}
    \begin{center}
        \vspace*{5cm}

        {\Huge\textbf{Geotechnical Simulations with Moose}}

        \vspace{5cm}
        %Upgrade Report

        \large{-- DRAFT --}

        \vfill

        \large{This whitepaper collects information and code \\
            for geotechnical simulations with the software Moose.}

        \vfill

        \textbf{Jörg Meier}

        \vspace{0.8cm}

        Gruner AG, Geotechnical Engineering\\
        Basel, Switzerland\\

        \vspace{1cm}
        16.10.2024\\

        \vspace{1cm}
        License: CC-BY-SA-4.0

    \end{center}
\end{titlepage}

\tableofcontents

\chapter{Introduction}
\label{chap:introduction}
The open-source, parallel finite element framework Moose
(\url{mooseframework.inl.gov}) can also be used to perform simulations that are
used in the geotechnical design practice. This document aims to collect options
and techniques that are suitable for creating such geotechnical simulations
with Moose.

In the geotechnical planning practice, finite element models are often used for
deformation prognosis. Furthermore, such FE models can help to identify complex
failure mechanisms. The requirements for these models are accordingly a
sufficiently good representation of the geotechnical structure including the
subsoil, the groundwater conditions, pre-existing structures and similar. These
FE models simulate the construction process over time. As this design process
is generally intended to take place in a controlled and quasi-static form, the
associated simulations are usually also transient and quasi-static. Therefore,
scope and main focus of this document are transient quasi-static simulations.

This document is primarily intended as a collection of information rather than
a textbook or step-by-step guide. As such, it is intended to be a 'living'
document that will be continually edited and updated.

This document is published under the
\href{https://creativecommons.org/licenses/by-sa/4.0/}{CC-BY-SA-4.0} license.
No warranties are given.

\chapter{Geometry}
\label{chap:geometry}
\section{General Considerations}
\label{geometry-general}

When modelling geotechnical problems, it is often desirable to
make a meaningful geometric abstraction to avoid time-consuming
modelling, but also to avoid inaccuracies due to oversimplification.
However, experience shows that the effort involved in creating such
a model geometry should not be underestimated and that the models
consist of a large number of individual parts and components.
Some of these objects have trivial shapes (e.g. rectangular surfaces
for a sheet pile wall segment), others can have non-trivial curvilinear
boundaries (e.g. geological bodies).
All of these objects must be shaped to fit each other (without gaps or
overlaps) and together form the FE model.

As the shape of these individual parts and components directly influences
the shape of the FE elements they are build of, further prerequisites must
be considered during geometrisation: For instance, as few ‘unfavourably
shaped’ (e.g. too acute-angled) FE elements as possible should result in
order to avoid numerical instabilities or load increments that are too
small. A common reason for acute-angled elements and large numbers of
elements are corner or end points that are close together and unfavourable
intersections of the objects. Another requirement for geometrisation
is that the number of FE elements should remain within a range that
enables acceptable calculation times.

Although Moose's ‘Mesh System’ provides a variety of tools for generating
FE meshes on the basis of input file commands, the creation of geometrically
complex models with these commands is often confusing and error-prone.
% https://mooseframework.inl.gov/syntax/Mesh/index.html

The mesh is therefore often created outside of the Mosse (e.g. using
\href{https://blender.org}{Blender}), saved as MSH-file and then imported
using the \codeword{FileMeshGenerator}.
The code required for this import in a Moose input file is shown in
listing \ref{FileMeshGenerator}.
% \href{https://mooseframework.inl.gov/source/meshgenerators/FileMeshGenerator.html}{FileMeshGenerator}

\begin{lstlisting}[caption={Read mesh from a file},label={FileMeshGenerator}]
[Mesh]
    [file]
        type = FileMeshGenerator
        file = "source.msh"
    []
    second_order = true
[]
\end{lstlisting}

(to be written)


\chapter{Model Elements}
\label{chap:elements}
(to be written)

\section{Volumes}


\section{Plates (surface element)}


\section{Interfaces (surface element)}


\section{Contact}


\section{Beams (line element)}


\section{EmbeddedBeams (line element)}


\section{Springs between nodes (NodeToNodeAnchors)}

https://mooseframework.inl.gov/mastodon/source/materials/LinearSpring.html


\section{FixedEndSprings}

https://mooseframework.inl.gov/source/nodalkernels/CoupledForceNodalKernel.html

https://onlinelibrary.wiley.com/doi/pdf/10.1002/pamm.202200045

\section{Pore Pressure Boundaries}


\chapter{Material Models}
\label{chap:materials}
\section{Preliminary remarks on the material models}
\label{chap:material-remarks}

The built-in library of material models suitable for modelling soil and rock is
currently very limited.

\section{NEML2}
\label{chap:material-NEML2}

Moose supports the use of materials defined by means of the external material
modeling library \href{https://github.com/reverendbedford/neml2}{NEML2}
(Messner and Hu). Details on the integration of NEML2 into Moose are not scope
of this document and can be found in the corresponding
\href{https://mooseframework.inl.gov/moose/modules/solid_mechanics/NEML2.html}{corresponding
    section of the Moose documentation}.

\section{Mohr-Coulomb (linear-elastic ideal-plastic)}
\label{chap:material-mohr-coulomb}

(to be written)


\chapter{General Model Setup}
\label{chap:setup}
\section{Strain calculation}
\label{chap:setup-strain}

In the
\href{https://mooseframework.inl.gov/modules/solid_mechanics/}{'solid\_mechanics'}
module, Moose supports three different types of
\href{https://mooseframework.inl.gov/modules/solid_mechanics/Strains.html}{strain}
calculation:
\begin{itemize}
  \item {Small Linearized Total Strain}
  \item {Incremental Small Strains}
  \item {Finite Large Strains}
\end{itemize}

\begin{lstlisting}[language=perl, caption={Setting up incremental small strains within the Physics/SolidMechanics block},label={setup-incremental-small-strains}]
[Physics]
  [SolidMechanics]
    [QuasiStatic]
      [./all]
        strain = SMALL
        incremental = true
        (*@{\raisebox{-1pt}[0pt][0pt]{$\vdots$}}@*)
      []
    []
  []
[]
\end{lstlisting}

\section{Stress calculation}
\label{chap:setup-stress}

\href{https://mooseframework.inl.gov/modules/solid_mechanics/Stresses.html}{Stresses.html}

(to be written)

\chapter{Initial Conditions}
\label{chap:initial}
\section{Initial conditions for Moose objects}
\label{chap:IC-moose-objects}

For definition of the initial (starting) conditions for the variables a Moose
simulation the \href{https://mooseframework.inl.gov/syntax/ICs}{ICs system} is
to be used.

(to be written)

\section{Gravitation and initial stress state}
\label{chap:IC-stress-state}

The initial stress state of geotechnical simulations is often non-trivial. This
is because the section of geosphere to be modelled often has a history of
millions of years under different loading conditions, including gravity and
tectonic processes. In addition, the Earth's surface is often non-horizontal,
so that stress trajectories near the surface are oriented accordingly.
Furthermore, the initial stress state may be anisotropic with a specific
orientation. Gravity and the initial stress state are normally taken into
account by a combination of the following elements:

\begin{itemize}
  \item Definition of appropriate displacement boundary conditions for the model (see
        \autoref{chap:model-configuration-boundary-fixities}).
  \item Activation of gravitational body force (see
        \autoref{chap:IC-stress-state-gravity}).
  \item Definition of eigenstrains from the initial stress field (e.g. using
        \href{https://mooseframework.inl.gov/source/materials/ComputeEigenstrainFromInitialStress.html}{ComputeEigenstrainFromInitialStress},
        see \autoref{chap:IC-stress-state-simple} and
        \autoref{chap:IC-stress-state-anisotropic}).
\end{itemize}

\importantbox{
  In simulations including pore water, ‘effectice stresses’ are to be used.
  This includes the definition of the initial stress field, where also
  effective stresses are to be given.
}

\subsection{Gravity}
\label{chap:IC-stress-state-gravity}

To consider gravity, the following aspects must be included in the model:
\begin{itemize}
  \item When compiling the MooseApp: As the effect of gravity is taken into account in
        interaction with the material density, the Moose module \codeword{MISC} must be
        active in addition to the Moose module \codeword{SOLID_MECHANICS}.
  \item Gravity must be activated in the input file. This is done by adding a special
        kernel of the type \codeword{Gravity} as shown in
        \autoref{initial-conditions-gravity}. The direction and strength of the
        gravitational field is to be defined here.
  \item The materials must be assigned a material density
        (\autoref{initial-conditions-density}).
\end{itemize}

\begin{lstlisting}[language=perl, float, caption={Gravity kernel in a Moose inut file},label={initial-conditions-gravity}]
[Kernels]
    [gravity]
        type = Gravity
        use_displaced_mesh = false
        variable = disp_z           # the displacement variable the gravity is associated with
        value = -9.81               # in m/s^2
    []
[]
\end{lstlisting}

\begin{lstlisting}[language=perl, float, caption={Assignment of a density to subdomain ‘block1’},label={initial-conditions-density}]
[Materials]
    [undrained_density]
      type = GenericConstantMaterial
      block = 'Block1'
      prop_names = density
      prop_values = 2500            # in kg/m^3
    []
[]
\end{lstlisting}

{\hfuzz=20pt
\subsection{Initial stress state using ComputeEigenstrainFromInitialStress}
}
\label{chap:IC-stress-state-simple}

In a uninfluenced homogeneous horizontal half-space (or a section thereof)
under the effect of gravity, the initial stress state corresponds to a
‘geostatic stress field’. This geostatic stress field can be represented
analytically for a stress point as follows:

\begin{equation}
  \sigma'_{ini,vert}=\sigma'_{ini,vert,h=0}+\gamma' h
\end{equation}
\begin{equation}
  \sigma'_{ini,horiz}=\sigma'_{ini,vert} K_0 \quad \text{with } \quad K_0 = 1-\sin\varphi
\end{equation}

In these equations, $\gamma'=\rho' g$ corresponds to the weight of the soil,
$h$ to the overburden height, $\sigma'_{ini,vert,h=0}$ to the effective
vertical stress at $h=0$ and $K_0$ to the earth pressure coefficient. This
earth pressure coefficient is usually estimated as a function of the angle of
internal friction $\varphi$ (simplified equation according to Jaky).

\autoref{initial-conditions-ComputeEigenstrainFromInitialStress} shows the
definition of a geostatic initial stress state using two functions
\codeword{ini_xx_yy} and \codeword{ini_zz} and their subsequent use in a
\codeword{ComputeEigenstrainFromInitialStress}.

\begin{lstlisting}[language=perl, float, caption={Definition of a geostatic initial stress state using ‘ComputeEigenstrainFromInitialStress’ },label={initial-conditions-ComputeEigenstrainFromInitialStress}]
[Functions]
    [ini_xx_yy]   # function describing the initial effective horizontal stress field
      type = ParsedFunction
      vars = 'sig_top   z_top   rho      g    k0 '
      vals = '-1.5      0       0.0025   10   0.3'
      value = '(sig_top - rho * g * (z_top - z)) * k0'
    []
    [ini_zz]      # function describing the initial effective vertical stress field
      type = ParsedFunction
      vars = 'sig_top   z_top   rho      g'
      vals = '-1.5      0       0.0025   10'
      value = '(sig_top - rho * g * (z_top - z))'
    []
[]
  
[Materials]
    [strain]
      type = ComputeIncrementalSmallStrain
      volumetric_locking_correction = false
      eigenstrain_names = ini_stress        # pointing to the initial stress field
    []
    [ini_stress]   # initial stress field (effective stresses)
      type = ComputeEigenstrainFromInitialStress
      eigenstrain_name = ini_stress
      initial_stress = 'ini_xx_yy 0 0   0 ini_xx_yy 0   0 0 ini_zz'
    []
[]
\end{lstlisting}

\subsection{Initial stress state for anisotropic stresses}
\label{chap:IC-stress-state-anisotropic}

Align the model to the orientation of the initial stresses

(to be written)

\subsection{Checking the initial stress state}
\label{chap:IC-check}

In geotechnics, the initial state including the initial stress state should be
in equilibrium. Therefore, in a first time step without any changes to the
model ‘nothing’ should happen (only neglectable deformations and changes in the
stress field).

This can be checked by introducing such a time step without any changes as a
first time step and assessing the system response (e.g. in Paraview).


\chapter{Construction Stages}
\label{chap:stages}
\section{General comments on construction stages}
\label{chap:stages-general}

The type of geotechnical simulation primarily discussed in this document aims
to model a construction process. Accordingly, such a simulation must reproduce
the various stages of construction over time. Since this document assumes that
Moose models are organised on the basis of subdomains
(\autoref{geometry-blocks-and-subdomains}), the state changes also refer
primarily to adjustments that affect entire subdomains.

The most common state changes in this context are:

\begin{itemize}
      \item Activating FE-elements (e.g. installation of sheet piles is usually modelled by
            activating the corresponding set ot FE-elements,
            \autoref{chap:stages-element-activation-deactivation})
      \item Deactivating FE-elements (e.g. excavation of a soil volume is usually modelled
            by deactivating the corresponding set of FE-elements,
            \autoref{chap:stages-element-activation-deactivation})
      \item Replacing FE-elements (e.g. installation of a slurry wall modelled via volume
            elements leads to the need to replace the soil volume with the volume
            representing the slurry wall, \autoref{chap:stages-element-replacement})
      \item Adjusting external loads (e.g. loads are used to model various influences. As
            these influences change over time, the corresponding loads have to be adjusted,
            \autoref{chap:stages-loads})
      \item Adjusting prestress (e.g. for props and stand anchors the prestress is to be
            adjusted, \autoref{chap:stages-prestress})
\end{itemize}

As a consequence of the Moose input file concept, these status adjustments must
be triggered at various points in an input file if Moose is used ‘out of the
box’. To provide a more centralised way of defining construction stages, the
\codeword{[Stages]} MooseApp code plugin linked below can be used:

\href{https://github.com/jmeier/moose-codeplugin-stages}{github.com/jmeier/moose-codeplugin-stages}

This document primarily describes the procedure for the definition of
construction stages, with the use of the \codeword{[Stages]} code plugin where
applicable. The procedure without this plugin is only mentioned in selected
places.

\section{Time steps without changes}
\label{chap:stages-without-changes}

Avoid time steps (and therefore stages) without changes to the model. Time
steps without changes may not converge, even if a convergence was found for the
previous time step - compared to which no changes were made (due to the
relative error being ‘high’).

\todoinline{elaborate on timesteps without changes}

\section{Abrupt versus gradual changes}
\label{chap:stages-gradual-changes}

For transient simulations, the speed at which changes are applied to the model
is usually important. When elements are deactivated, the self-weight of these
elements is removed, all loads and stresses transmitted by these elements must
be redistributed, and so on. If many elements are deactivated, the resulting
changes in the stress field will be correspondingly large. In a not purely
linear-elastic model, comprehensive changes can lead to numerical instability,
resulting in slow or no convergence.

For geotechnical problems, on the other hand, changes in reality are usually
gradual.

(to be written)

\section{Activating and deactivating FE-elements}
\label{chap:stages-element-activation-deactivation}

%check: https://caeassistant.com/blog/abaqus-element-deletion-criteria/

As mentioned in \autoref{geometry-blocks-and-subdomains}, Moose models are
organized and managed by means of subdomains. By assigning FE elements and
materials to a subdomain, for example, the behaviour of these FE elements is
defined. However, Moose does not offer the option of activating or deactivating
FE elements or subdomains during a simulation run. However, Moose allows FE
elements to be reassigned from one subdomain to another during a simulation run
using the
\href{https://mooseframework.inl.gov/syntax/MeshModifiers/index.html}{MeshModifiers}
system.

On the other hand, Moose allows for subdomains to exist without kernels,
variables, materials and other Moose objects, but for FE elements to be present
in these subdomains. FE-elements in those subdomains ‘without physics’ will
simply be ignored my Moose.

Activation and deactivation of FE elements can thus be modelled by combining
the option to have subdomains ‘without physics’ and the option to reassign FE
elements to another subdomain. For deactivation FE elements are moved from a
subdomain ‘with physics’ to a subdomain ‘without physics’. For activation vice
versa.

ToDo:
\begin{itemize}
      \item Activation / Deactivation with [Stages]
      \item Ramp up / ramp down material properties, \autoref{chap:stages-gradual-changes}
\end{itemize}

\section{\todomarker Replacing FE-elements}
\label{chap:stages-element-replacement}

(to be written)

\section{\todomarker Adjusting external loads}
\label{chap:stages-loads}

(to be written)

\section{\todomarker Adjusting prestress}
\label{chap:stages-prestress}

(to be written)

\chapter{Postprocessing}
\label{chap:postprocessing}
\section{Paraview}
\label{chap:postprocessing-paraview}

Directly use the ExodusII-files produced by Moose.

(to be written)

\section{Blender using the BVtkNodes addon}
\label{chap:postprocessing-blender}

The \href{https://github.com/tkeskita/BVtkNodes}{BVtkNodes addon} for Blender
that wraps the Visualization Toolkit (VTK) library for scientific visualization
in Blender.

(to be written)

\section{Directly accessing ExodusII files}
\label{chap:postprocessing-ExodusII}

The primary output format of Moose is ExodusII. Accessing the data directly in
this format is therefore a logical step. To work with ExodusII-files under
Windows, one might want to compile the corresponding code in the
\href{https://github.com/sandialabs/seacas}{seacas-repo} under Windows. The
step by step instructions on how to compile ExodusII for windows can be found
here: \href{https://github.com/sandialabs/seacas/discussions/375}{seacas\#375}.

(to be written)


% \appendix
% \chapter{Quick-Start}
% \label{app:quickblender}
% \input{Appendices/quickblender.tex}

\end{document}